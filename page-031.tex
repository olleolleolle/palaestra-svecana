\newpage

\scan{031}

\newpage

% Translation below

% fötterna och kroppen rätt och med gått förstånd weet att regera /
% der med att klingan hennes ämbete och wärken dess bättre kan
% fórrätta / hwilket och fodras / att man i ophäfning den ene foten /
% kroppen hwijlar så länge på den andre utan någon Slancering, och
% att den ene foten är altijd befrijat af kroppen / och blifwer
% obeswärat och oförhindrat / så blifwer man sin kropp och Klinga
% mycket mechtigare.

feet and body right and with good sense knows to rule / that the 
blade can execute its office\sidenote{The original uses ``ämbete'', literally office
in the political sense} and work / which requires / that one in the
lifting of one foot / rests the body on the other without
counter-thrust\sidenote{Original uses ``slancering'', which apparently
means just that, but makes no sense in ths context?}, and one foot is
always freed from the body, and remains untroubled and unencumbered /
so your body and Blade become much mightier.

% Hwar hoos är och att merkia / att uti passering i fortsättning
% wenstra foten / att man intet tillijka kommer fram med wänstra
% Axelen utan den högre / hwilket den wänstra accompagnerar, måtte
% fördenskull föhra tån ut på wänstre foten wäl / så att man bättre
% kroppen sinkos / och blifwer man med den samma intet så långt
% tillbaka ; kan man altså bruka Klingans styrkia fast bättre och med
% stöten ( emedan med kroppen på sådant sätt mer till pendicolar ) har
% så långt som han har kunnat giort med högre foten.

In which you should note, that in passing in continuing the left foot
/ that you do not bring forward the left Shoulder but the right /
which accompanies the left, must still extend the toe on the left foot
well, to better feint\sidenote{This was under the mis-reding of
'sinkos' as 'fintos', this meaning is not entirely clear, if could be
``delayed'', it could be ``bent''} with the body / and in the same not become too
far back; you can thus use the strong of the Blade better and in with
the thrust ( since the body in such a way further [pendicolerar???])
have as far as you can possibly do with the right foot.

% Enteligen att man och i paserande som tillförende korteligen är
% omrört / brede wijd sin Fiendens Klinga | han ligge med henne stilla
% eller intet Continuera och henne altijd föllia skall; Så håller man
% henne i med säkerheet effterfölliande proportion, att man när han i
% gvardian är öfer eller jämpte hans swaga / hwar effter han ligger
% högt  eller lågt / gå in med en stadig Arm / at man honom med
% Klingan når / och som man kommer diuper och diupare in / ju mer 
% strecker man sig der meed; Men ändå att man intet mycket berör henne
% / till des man kommer med korset / der man tillförende war med udden
% / och till hans kropp Continuera och stöta / i dett man lijkwäl hans
% Klinga / der med man sig intet alt för mycket blottar / hwarken
% under eller öfwer sig  / eller till sijdan / intet mer än den
% fortgång som wärkar utaf sig sielf ; Effter samma proportion och
% mått skall man och rätta udden till hans blott.

Finally that you in passing as is shortly mentioned / broadly by your
Enemy's Blade | he keeps it steady or does not Continue you should
always follow it; Thus do you hold it with safety the following
proportion, that when he is in guard over or beside his weak / when he
is high pr low / enter with a steady Arm, that you reach him with your
Blade / and you get deeper and deeper in / the more you extend; But
not so that you touch it much / until you reach the hilt / where you
just were with the point / and to his body Continue and thrust / in
that equally his Blade / thus do you not open too much / neither below
nor above / or to the side / no more than the progress that works on
its own; In the same proportion and measure shall one also correct the
point to his opening.

% Men går han i passaden med sin Klinga öfwer eller under ifrån din
% Klinga / måtte man i dett samma ögnebleck föllia hans Klinga / att
% lijka som bägge Klingorna woro tillhopabundne / så att den ene
% drager den andra med sig / med en raak Arm / och med fremste ledens
% rörelse med den högre handen.

But if he in his parry brings his Blade above or below your Blade /
you must in that instant follow his Blade / as if the two Blades were
tied together / so that one drags the other with it / iwth a straight
Arm / and the movement of foremost joint\sidenote{This is most
probably the wrist, rather than a finger joint} of the right hand.

% Men wore din Fiendes mouvement små och geswinna Cavationer, eller
% muterade / blifwer man med Klingan der med man sig icke sielfwer
% disordonerar jämpt för honom / och Avancerar med udden till hans
% kors till des man får mens' till stöten.

But if your Enemy's movements are small and quick beats, or mutated /
your Blade stays where you are not disadvantaging yourself, compared
to him, and Advance the point to his hilt until you get measure for
the thrust.

% In Summa, att stöta  pie fermo är en art att fechta / och passera en
% annan / och den som har bägge wettenskaperne kan taga hwilkerdera
% han bäst befinner sig vara skickader och benägen till; En till
% passering hör en stor skickeligheet / stadigheet / geswindhet /
% gottt jugement och resolution, whlket intet finnes hos whar och en, 

To sum up, to thrust a pie fermo, is an art of fencing / and passing
another / and the one who has both these sciences can use whichever is
best suited to be skilled and partial to. To passing is a great skill
/ stability / quickness / good judgement and resolve, which is not
foudn in everyone.