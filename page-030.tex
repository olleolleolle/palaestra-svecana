\newpage

\scan{030}

\newpage

% Translation below
% Fötterna går i dett samma mer unierat. hwilken samandrechtigheet
% förorsakar en särdeles styrka och geschwindigheet i sielfwa
% förrättandet/ och kan man i ingången en effect i den andra
% beqwemligen förändra / altså att Fienden med besfär kan sig
% defendera; Blifwer honom och tijden i månge ränker att giöra, eller
% af dens andras förtagande rätt att judicera betagen / och oansedt
% occasion löper snart förbij / och när man har passerat hans udd är
% man för hans stöt säker. 

Feet go immediately more unified, which altogether causes a special
strength and swiftness in the execution itself, and can in the closing
comfortably cause an effect in the other, that is the Enemy can only
with difficulty defend. This gives him little time to jduge and
consider, and his moment is soon past, and when you have passed his
point you are safe from his thrust.

% Men i stöten a pie fermo händer sig offta / att man offta trädt så
% diupt in / eller och Fienden i dett samma har gåt fram för sig / har
% han råkat så diupt in i mens' att man intet har kunnat komma
% tillbaka / och uti retireingen blifwit stötter ; I hwilket fall är
% gott / att man går fort in till sin Fiendes kropp / emedan den
% största faran är när man kommer till mens' men när udden är
% passerat, och man förföllier till hans kropp / kommer man till honom
% för än han kan retirera sin Klinga.

But in a thrust a pie fermo it often happens, that one has stepped so
deeply, or and the Enemy ahs in that moment stepped forward, he so
deep into measure that one has been unable to return, and in the
retreat receive a thrust; In any case it is good, that one quickly
steps in towards one's Enemy's body, since the greatest danger once in
measure but the point is passed, and one follows to his body, you come
to him before he can retreat his Blade.

% Dett händer sig att man har med udden passerat, och stöt att Fienden
% drager Armen i det samma tillbaka och laederar ; En dett är dens
% faute som passerade, att han intet Continuerade till sin Fiendes
% kropp eller intet tagit tempo wäl; En om han i dett samma / när
% Fienden förer Klingan fram för sig / eller i dett han är med deffeta
% occuperat eller går utur presenza, kan han intet i den samma tijden
% / när man passerar på sådant sätt retirera eller dra tillbaka.

It sometimes happens that one has passed with the point, and thrust
that the Enemy in that moment pulls his Arm back and [laederar???]; In
this it the [faut???] of the one who passed, taht he did not Continued
to his Enemy's body, or did not take tempo well; Only if he in the
same, when the Enemy moves the BLade in front of him, or in that he is
with [deffeta] occupied, or goes out of presenza, he cannot at the
same time, when one has passed in such manner, retreat or pull back.

% Är och nödvändigt när man passerar att man altijd föllier Fiendens
% Klinga / och altijd blifwer hos den samma att man säkert der hos
% fortgå kan.

It is also necessary when passing to always follow the Enemy's Blade,
and always stay in a position taht one can always proceed.

% Men det äro några andra / när de allredan aldeles hafwer passeeat,
% sig retirera och söka att stöta / hwilket lättigare låter sig giöra
% med en stackot än med en lång Klinga; der emot man måtte wetta / att
% Klingan hon må wara lång eller stackot / så lär dock den samme som
% passerar, går och wäl sutten / och då han skulle komma med Klingan
% så diupt med en scurso ingå till sin Fiendes kropp; En uti passering
% kan man giöra åtskillige ting / först med kroppe att man stöter sin
% Fiende / och honom der med disordonerar, sedan taga i hans fäste /
% så kan man och om man intet har råkat med passaden, sättia sin
% foot bak om sin Fiendes och med ringning kasta honom till marken.

But there are some, that when they have already passed, that retreat
and seek to thrust, which is easier done with a
short\sidenote{Original has ``stackot'', from context 'short' has been
inferred} than a long Blade; on the other hand one must know, that the
Blade may be long or short, but he who passes, is well situated, and
then he should come with the Blade so deep with a [scurso???] step in
to his Enemy's body; In passing you can do a multitude of things,
firstly with the body, to thrust one's Enemy, and him thereby
[disordonerar???], then grab his hilt, and one can also if one did not
land a thrust with the passage, place one's foot behind the enemy's
and with a ringing\{original has ``ringning'', literally ring-forming
or ringing (like a bell)} throw him to the ground.

% I lijka måtto är den samma som passerar, i all tillfällen mycket
% ferdigare och beredsammare än den andra som är med diffesa
% occuperat, och uti den faran som han befinner sig uti blifwer
% confunderat.

In equal measure is the one who passes, in all moments much more ready
than the other who is with [diffesa???] occupied, and in the danger he
finds himself become confused.

% Den som wäl weet och kan passera, förer Klingan med mera qvis terta,
% twingar sin Fiende mycket bättre / är mycket säkrare på sin sak /
% särdeles när han 

The one who well knows and can pass, holds the blade with more [qvis
terta???], forcing his Enemy much better, is much surer of his
intention, especially when he
