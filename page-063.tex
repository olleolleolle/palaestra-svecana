\newpage

\scan{063}

\newpage

% Translation below

% Warder altså här med för denna gången förklarat / hwad sielfwe
% Fundamentene / Genearl Regles / såsåm mig tyckes till denne Konsten
% uthi Fächtning wara behörige och nödwändige / förmondandes icke
% annorlunda / än den som hafwer lust och någon wettenskap aff denne
% exercitien, lärer kunna här aff någon meera förnöjelse och
% eftertänkande. Hwarföre till större uplysznin will iag här uppå låta
% föllia någre få lektioner med des nödige Kopparstycken / som iag
% förhoppas til detta / ware det ene med det andre nogsamt och till
% fyllnadt: En der den högste Gud skulle behafa att förlängia min
% Lifstijd med de tankar som iag härutinnan drager /
Thus is here for this time explained, what the Fundaments, General
Rules, as I think to this Art of Fencing being required and
sufficient, presuming no different, that the one who have the
inclination and some knowledge of this exercise, should here to some
more enjoyment and contemplpation. Wherefore to larger enlightenment I
will here upon let follow some few lessions with the required
Chalcographs, that I hope to this, be with one and the other carefully
and with fulfillmen. If it would please the highest God to lengthen my
Lifetime wit the thoughts that I herein express,
%                                                    så blifwer mitt
% opsåth at låta ännu ett annat wärk utgå uppå wårt tungemåhl / som
% intet af allom kan hända låter tillförende wara lett eller hort /
% mindre practicerat, såsom der utinnan något mera än wanligt uti
% denna konsten blifwer framstält och upwist: En iag måtte bekänna
% sanningen / att iag aldrig någon mer än en stor och förnem Herre har
% i Swerige / hos den iag hafft den nåden på månge åhrs tijd med min
% tienst få opwachta / hwilken hafwer gifwit mig stor anledning / nmed
% siff fullmogne arbete och högre discurser / att den högste
% wettenskapen af denne Konsten och Exercitis nogare efftertänka /
my intent will be to let another work be done in our tounge, for as
everything can happen easy or hard, less practiced, like before this
somewhat more than common in this art be exposed and exhibited. Still
I must confess the truth, that I have never had a grander and nobler
Lord in Sweden, for whom I have had the grace to court with my service
for many years, which has given me a grand reason, with his
well-developed work and higher discourse, that the highest knowledge
of this Art and Exercise more closely contemplate,
% practicera och under dagen föra / hwilken kallas Caminera; Det är at
% så snart man hafwer lagt werjan i handen / går man på sin Fiende lös
% / och utan något tillstånd med Resolution läderar / Dock at bekomma
% denne fullkomligheten kräfwer mycken flijt och monga åhrs arbete /
% wharföre måste man först wara uti Lågerfechtning ferm och wäl öfwat
% / så att man har bekommit en stadighet uti alle leder / en frij och
% otwungen hand / en suptil känszlo / och att man i en rätt tijd kan
% cedera / och sin effect med en hast förändra.
practice and during the day do / which is called
[Caminera]\sidenote{This is untranslated from the original, at this
point in time, I simply do not know what this word means.}. It is so
soon after one has laid the rapier in hand, and let loose on one's
Enemy, and without any permission with Resolution injures. Nonetheless
to become this completeness requires myck diligence and many years
work, wherefore one must first in Lower fencing firmly and well
practiced, so that one gains stability in all joints, a free and
unforced hand, a subtle feeling, and one in the corret moment can
retreat, and one's effect with haste change.
%                                                Derrföre will iag
% korteligen opwijsa whad fördel den hafwer / som med Resolution
% angrijper / emot den stilla ligger. Hwarutinnan i begynnelse är till
% Confiderere / att den samma som hafwer lägrat sig / och sedan wille
% sig först röra / det han för sin tyngd skuli altijd är långsammare
% än den som alladedan är i mo?u; En den som hafwer fått neder både
% fötterne / kan icke utan med twå tempo någon dera röra / och
% fortföra / det är / den ena uti opphöjningen och den andra i
% nedersättningen; Der emot den samma som är uti gången eller fartn
% hafwer altijd en fot i Högden /hwilken han uthan twifel förr låter
% hafwa satt neder / förr än den andre sin uplyft / hwilket är icke
% ett ringa förträff / att man allareda hafwer fullgiort sin werkning
% / medan den andre nu först begynner.
Therefore I want to briefly demonstrate what advantage he has, who
with Resolution attacks, against the one who is still. Wherein in the
beginning is Confidence, that the one who has stayed himself, and then
wanted to be the first to move, due to his weight will always be
slower than the one who is already in motion\sidenote{This is a bit of
a guess, the original is a bit hard to read}. Since the one who has
planted both feet, cannot move either in less than two tempos, and
complete, this is, the first in the lifting and the second in the
lowering. Conversely, the one who is walking or at speed always has
one foot Elevated, which he without doubt quicker can put down, before
the other one has lifted hius, which is not a small advantage, hat one
has already completed one's effect, while the other now just starts.
%                                      Här hoos gifwer och den samma
% som hafwer lägrat sig / åth den andre meer lägenheet och tillfälle /
% honom uti hans Guardia at betrachta och judicera / huru kan man
% något tillbringa honom / som man seer utan uppehåll gå på sig löss?
% En förr än han något resolverar och omtänkt sigh, lär occasion
% allredo wara förbij: Altså är intet twifel at de tempo som wijsa sig
% då / äre mycket expediter aff dem samma som äro i rörelsen / än aff
% den som hafwer satt Fötterne neder / kunna blifwa tagne; En i det
% samma han williatr röra sig at taga tempo , är det redan hoos den
% andre fullbracht och förnij flugit / och såsom detta är för sendt
% till wåga(???) kommit
Here thus gives the one who has stilled, to the other more opportunity
and occasion, him in his Guard to observe and judge, how one can
something bring him, that one sees without pause go loose? As he
resolve and considers, the occasion will already be lost. It is thus
no doubt that the tempo that show themselves, are much more expedient
to them who are in motion, than by the one who have Feet firmly down,
taken.  In that the moment he wills to move to take temp, it is
laready in the other brought to completion and flown past, and like
this it is too late.\sidenote{The tail end of this sentence is a bit
hard to read in the original.}
