\newpage

\scan{024}

\newpage

% Translation below

% Ett sådant manier att Fintera ( när det är i flijtig öfning bracht)
% kan man alla Lectioner lätteligen bemöta och utan möda stöta. Denna
% ahrt att Fintera är mycket förfärlig emot dem som giöra
% Chiamater. När nu Fienden parerar effetr din Fint, går du med din
% udd i en linea ända fram till hans kropp / på det du (för än han sin
% motion i parerande hafwer ändat eller fått bryta mens') med din stöt
% hafwer råkat / emädan du geck med Finten fort / lähr du och förr
% blifwa färdig / än han får taga det andra tepo eller bryta mesuram,
% byter han (i dett samma Du giör Finten) Mens' motte du intet fara
% fort med din Fint; En han är så långt ifrå dig / utan söka på nytt
% att string' der emot när han intet retirerar sig / en häller will
% parera effter din Fint, utan låter gå din Klinga så långt in / att
% du kommer med din styrka emot hans swaga; Måtte du i ett ögnableck
% fullföra din stöt ända fram / till hans högra Axel / så lär han
% intet kuna bringa din Klinga utur wägen / utan blifwer af dig
% råkat. I sista??? måtto lär du och förbringa honom till att parera,
% när du tillförende hafwer råkat på en ohrt / och sedan giör Finten
% på samma ställe.

Such a manner  of Feinting ( when it is diligently practiced) can one
every Lesson easily meet and without effort thrust. This art of
Feinting is very terrifying against those who do Chiamater. When the
Enemy parries after your Feint, you go with your point in a line all
the way to his body / in that you ( before he has finished his
parrying motion or broken measure ) with your thrust have reached /
since you went with the Feint quickly / you should sooner be done /
than he can take a second tempo or break measure. If he breaks (in the
moment You make the Feint) measure, you must not proceed quickly with
your Feint. Since he is so far away from you / you should seek to bind
again when he is not retiring / or even parry after your Feint, you
should let your Blade go so deep that your strong comes against his
weak; Now you must in the blink of an eye complete your thrust all the
way / to his right Shoulder / he will be unable to move your Blade out
of the way / but will be reached by you. In the last measure you will
likely make his parry fail, when you have first reached one point /
and then Feints to the same place.

% När du på detta sätt som ofwantaldt  / Finterne blifwa giorde med
% porterer Klinga / och stöterna blifwa giorde till Finter, Kroppen
% altijd med Avantage af styrkan / och dett swaga får offeta (som af
% sin Fiendes han förorsakas kunde) försåtrat / äro Finterne
% fullkomligen / och äro de ett af dett fullkomligste och förnehmste
% uti Fechtare Konsten/ blifaw så offta man ingen fullkomlig mens' har
% till stöten. Item[???} de falska blotten och när Fienden har satt
% sig före att taga en stööt emot / men i dett samma råka tillijka /
% hwar utaf din Fiende blifwer ( i det man förändrar sitt arbete)
% bedragen.

When you in teh manner described above / Feints are done with
right-sided Blade / and thrusts anre made into Feints, Body always
with the Advantage of the strong / and the weak may frequently ( as by
his Enemy's can be forsaken ) be sacrificed / the Feints are made
complete / and is one of the most complete and noble in the Fencer's
Art / made so frequently when one no full measure have for the
thrust. Item[???]\sidenote{Check possible translations...} the false
openings and when the Enemy has put his mind to take a thrust / but in
the same also reaches / whereby your Enemy is (in that you change your
work) betrayed.

% Hwad anbelangar de Finter, der Fiendens Kling blifwer frij ock icke
% röres / äro sådana icke heller till att förkasta/ när man hafwer ett
% gott tempo för sig; hwarutinnan sig befinner twenne slag / det ena
% utan/ dett andra innan till: hwilka på fölliande sätt gjorde
% blifwa; Utan till / giöres han med en span eller Qvarter jern / rätt
% under hans kors / på dett man när Fienden parerar, hastigare kan
% cavera rätt till hans högra Axel i Tertia eller i Qvarta.

As concerns the Feints, where your Enemy's Blade is free and isn't
touched / those are not to be dismissed / when you have a good tempo;
wherein exists two kinds / one without / the other within; who are
done in the following manner. Without is done with a span or a Fourth
steel / right under his cross / so that when teh Enemy parries, quicker
can beat to his right Shoulder in Third or Fourth.

% Den Finten som skeer innan till i Tertia eller i Qvarta emot ett
% långt läger / wijd pass ett qvarter långt ifrån sin Wärja / att
% korset i en jämbd linea utan jern / der du din Kling ( är din Fiende
% parerar ) må hafwa frij / och fördenskull hastigare kan komma till
% att cavera.

The Feint than is done within in Third or Fourth against a long
position / to pass far from one's Rapier / the cross in an even line
without steel / so taht your Blade (when your Enemy parries) may have
free / and thus quicker can come to a beat.

% Dessa blifwa och som de förrige giorde / när man har formerat
% cotrapostur emot de medlare / och under postur i samma högden / och
% något diupare med Klingan i en jämbd linea åht sitt weka lijff; Men
% på både Finterna så wäl utan / som innan till / blifwer a [ pie??? ]
% fermo stötter / det wari då ett tempo till passaden wore stor nog;
% Och blifwer offta under sig sielfwa / och offta med de föregågne
% dublerat; Hwilket tienar der till / att man dess snarare bringar sin

These are made as the previous / when you have formed contrapostur
agains the [medlare???]\sidenote{Again, check possible meanings, this
is unlikely to be a negotiator or broker} and under the posture in the
same height / and somewhat deeper with the Blade ni an even line
towards one's weaker life\sidenote{Side???}; But for both Feints,
without as well as within; is a [pie???] firmly thrusted / there is
then enough tempo for the passage; And frequenly it is in itself
doubled; Which serves to much quicker bring your