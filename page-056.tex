\newpage

\scan{056}

\newpage

% Translation below
% Men som till Exempel/ när du derstädes emot en hög eller lågh
% gvard' med den vantagio aff Contrapostur Finetrade eller du hade
% stött / han i dett samma redan till sin wänstra sinkos hade parerat,
% och han gifwer sig öfwer sitt kors en angulum eller ett blott/
But for example, when you against against a high or low guard with the
advantage given by Contrapostur Feinted or you had thrusted, he has
already low to the left parried, and he gives over his cross an angle
or an opening
% Cederar du honom med Klingan i dett han begynner att parera, och
% låter din linie fara / och giör en angulum der emot / så lärer ( i
% dett han finner ingen resistance) falla mer utur presenza: Och
% särdeles/ när du med kroppen som sig bör abaisserar/ handen wänder
% du i förrige situ hos hans swaga/ emot en öfwer / i en heel emot en
% medel/ i en half secunda, emot en under-gvardia i tertia, och stöter
% i det samma a pie fermo eller med passaden öfwer hans kors uti en
% angulo som tilförende sagt är; Udden föres wäl opföre till hans
% bröst / så lär han hafwa swårt att parera.
If you cede him with the Blade in that he begins to parry, and let
your line flee, and make an angle against it, you should (in taht he
finds no resistance) fall more out of presence. And especially, when
you lower your body as you should, you turn your hand into the former
position of his weak, aganist an over, in a whole against the middle,
in a half second, against an under-guard in third, and thrust in the
same a pie ferm or with the passage over his cross in an angle as
previously said. The point is taken well up his breast, he should have
difficulty to parry.

% Man stöter och ibland secunda med kropsens skränkning / när du har
% försummat de Tempo utan till att parera, så stijg geschwindt med din
% wänstra fot, det nehrutaf till hans högre på honom in / och gack
% med din Klinga utan till hans udd / så lär du hoonom derstädes åter
% igen discoopera; Din wenstra Axel lätt komma så långt fram / att du
% kan taga wijd hans kors om så behöfwas; eller wore af nöden / och
% att kroppen aldeles blifwer halberat / då kant du honom utan til
% öfwer eller under Klingan med secunda lädera.
We thrust occasionally in second with the bending of the body, and
when you have neglected the Tempo to parry on the outside, step
quickly with your left foot, near his right foot close to him, and go
with your Blade on the outside of his point, you should him thus again
disoccupy him. Your left Shoulder easily gets so far forward, taht you
can grasp by his cross if it is needed, or is necessary, and that the
body is completely halved, you can him on the outside above or below
the Blade with second injure.

% Elliest kan denna Parade brukas emot slascerte och resolute stöter /
% der man intet har at befruchta sig det Fienden lär förändra sin
% effect; I det parera honom hans stöt med styrkan af din Klinga/ och
% träd i det samma utan til med din högre Fot a traverso på honnom in/
% så at din öfwerdel af kroppe blifwer sktänkt och fast halberat / och
% din Arm tillijka med Klingan i en linea; Armen blifwer något
% förkortat / passera i det samma geschwindt / och stöt per secunda,
% under hans Klinga fort.

Also this parry can be used against slascerte\sidenote{translation
unknown, cuts?} and firm thrusts, where you have no fear that the
Enemy will change his effect. In that parry his thrust wit the strong
of your Blade, and step in the same on the outside with your right
foot a traverso in, so that your upper body becomes bent and firmly
halved, and your Arm withthe BLade in a line. The Arm is slightly
shortened, pass in the same quickly and thrust in second under his
Blade quickly.

% Secunda blifwer och stötter med försänkt kropu / när du med honom i
% en öfwer gvard af Klingan i en alpari ligger och i din fortgång /
% för hans genomgång / måste förswara och beskydda dig / i det och
% när han utan till med styrkan i din swaga går till ditt Ansichte /
% stöter du secunda a pie fermo under till hans högre sijda.

Second is thrusted with loweered body, when you with him in a high
guard with the BLade in a nominal\sidenote{Apparently so, although
there's probably a deeper meaning here} lies and in your proceeding,
must defend and protect yourself, in that and when he on the outside
with the strong in your weak goes to your Face, you thrust second a
pie fermo under to his right side.

% Enteligen när han din Klinga utan till åth sin högra parerar eller
% uthrycker/ williandes stöta öfwer din Klinga / eller tillijka med
% ett insteg / med sin i din halfwa klinga indela så Coopera dig i dett
% samma med secunda / och stig der hos med sträckter kropp wäl fram
% / der med förbij / passera i dett samma innan till du kommer hans
% udd förbij / passera i dett samma innan till / och stöt med secunda
% tillijka med Cavation i ett arbete till hans öfwerdehl: Din udd låt
% i Coopering intet förhägia / elliest kunde han tryckia din Klinga
% under sig / och stöttia henne / i det han går igenom / och en qvart
% lijka som wriden innan till din kropp / stöta fort / altså och
% innan till mutatis mutandis. Hwad till de sista de underblotten
% denna ohrten anbelangar / blifwer den på ett sätt eller occasion
% med den uthwärtes under Klingan tagen / och när Fienden ickge
% gifwer den enar / tager du den andra.

Finally when he your Blade outside to his right parries or presses,
intending to thrust over your Blade, or with a step in, with is in
your half blade partition, so occupy in the same second, and step with
it with extended body well forwards, thus past, pass in the same on
the inside so you will pass his point, pass in the same on the inside,
and thrust in second equally with a Void in a single work to his upper
half. Your point let in occupation not overhang, otherwise he could
press your Blade under his, and thrust it, in that ge goes through,
and a quart like turned inside to your body, thrust quickly, thus and
on the inside {\it mutatis mutandis}. What in the end this place the lower
openings concerned, they are in a fashion or occasion withthe outside
under Blade taken, and when the Enemy doesn't give the one, you take
the other.

\subchaptitle{The Thrust in Third}
% Föllier altså Tertia hwilken emot en hög gvard utan til / öfwer
% Klingan wäl stråkter, der uti man och gifwer mindre blott, med
% halfwa kroppen blifwer

Here folllows Third, which against a high outside guard, over the BLde
well stretched, there in one gives fewer openings, with half the body is
