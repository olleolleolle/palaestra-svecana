\newpage

\scan{067}

\newpage

% Translation below
% \chaptitle{Emot ett långt Läger / i Secunda Tertia eller Qvarta,
% derutinnan din Adversarius ligger stilla. Ibland hwilka denne förste
% Figuren, framter en Stringering uti Qvarta}
\chaptitle{Against a long Position, in Second Third or Fourth, wherein your
Adversary is still. Among which this first Figure, cause a Binding in Fourth}

\exercise
% Ligger din wederpart långt med en raak Klinga så string: honom innan
% till med en halff Qvart (Fig 1) ligger han stilla / så lyfft i dett
% samma din högre fot till then enga Mensuren, sätt honom med en hast
% neder/ och stöt per Qvartam med uddens försänkning innan till hans
% kropp- a pie ferm fig 2. Men emot en Secunda måtte du med Korset gå
% någon högre/ hwar med Ansichtet blifwer bettre förswarat. (Fig 2.)
If your opponent is long with a straight Blade so bind him on the
inside with a half Fourth (Fig 1) if he is still, then lift in the
same your right foot to the narrow Measure, set it with haste down,
and thrust in Fourth with lowered point inside to his body a pie fermo
(fig 2), But against a Second you must with the Cross go somewhat
higher, wherein the Face is better defended (Fig 2).

\exercise
% Ligger han med udden öfwer sig/ och Korset neder/ så string. honom
% innan till/ disordonera i hast derstädes hans Klinga med en
% battute, der med han i din stöt intet kan angulera. dok utan fotens
% rörelse; gör battuten med din styrkia i hans swaga/ så har han dess
% bettre force. stöt der på Qvartam innan till hans Kropp a pie fermo
% (Fig. 2.)
If he is positioned with the point above him and the Cross below, then
bind him on the inside, disorganise with haste then his Blade with a
beat, whereby he in your thrust cannot angle. Without the foot's
movement, do the beat with your string in his weak, it will have more
force. Thrust then in Fourth inside to his Body a pie fermo (Fig 2).


\exercise
% Ligger han med udden i högden/ och är innan till blott/ så gå med
% din udd öfwer till hans/ will han då intet movera sig/ så stöt
% Qvartam med ophögt kors och försänkter udd till hans inwerss kropp;
% Men låg han högre med udden/ så gå honom men din swaga i hans halhwa
% styrka/ emedan han är med sin udd utur presensa så är dett nog/ att
% du äst för hans uddz infall förwarat ( Fig. 2. )
If he is with the point high, and is on the inside open, go with your
point over his, if he is not moving, then thrust in Fourth with
elevated cross and lowered point to his inverted body. But were he
higher wit the point, then go with your weak in his half strong, since
he is with his point out of presence this is enough, taht you are for
his point's attack defended (Fig 2).
