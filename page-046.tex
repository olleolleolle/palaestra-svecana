\newpage

\scan{046}

\newpage

% Translation below
% Såsom och de samme som med deras Klinga söka dens andres Klinga
% rijwa sin kosz / och pröfwa att få henne utur handen; hwar emot
% String' ähr och god / serdeles / när man går med udden temmeligen
% högt så kan han intet wäl få swagan; Att man och icke heller håller
% Klingan alt för hårdt; En håller att man sätter sig med sin Klinga
% emot hans / utan bruka sig Cavationorna, så lähr man lätteligen
% l{\ae}dera honom. I lijka motto emot dem / som taga Wárjan i både
% henderna / och söka der igenom deste hårdare at battera dens andres
% Klinga / hwar på de tänkia att stöta; Här emot är at bruka
% Cavationer Chiameter, och finter utan Jern. Endteligen / om Fienden
% förer en stöt til ditt högre lår / så sätt honom med en hast
% tillbaka och halt din udd longt till hans Ansichte eller öfwerlijf /
% så är hans stöt fåfäng / och lär löpa sig sielf på udden / du kant
% och der emot ligera.

Similarly those who with their Blade seek the other's Blade to tear
away / and tries to get it out of the hand. Whom against Binding is
good, especially, when you go with the point fairly high so hat he
cannot get the weak. Also that one does not hold the Blade too
hard. Do not hold your BLade steady against his, but use Disengages,
you shoudl easier injure him. In equal measure againt those, who take
the rapier in both hands, and seek thus to harder beat the other's
Blade, whereafter they think to thrust. Against this is to ise
Disengage Schemes, and feints without Iron. Finally, if the enemy
brings a thrust to your right thigh, quickly pull it back and hold
your pont long to his Face or upper body / thus is his thrust folly /
and he should run himself onto the point / you can also against it bind.

\chap{Defence against each and every position in particular}

% När du prima wäl Formera wilt / så statt med hel eller med
% förskränkt Kropp / och sträck Klingan wäl fram, ut / hennes högd
% jempt med Hufwudet / Klingan och udden rakt ända fram / så mycket
% möjeligit är; En elliest af naturen är hon till Jorden benägen / der
% med att Fiende icke utan till kan få insteg der hon rå som swagast /
% måste och wäl wara betäckt; Under delen af Kroppen tag wäl
% sinkosz /  fötterne til hopa / och bucka di wäl / på det din Fiende
% intet under kan råka dig / ofwan effter åst du af din Klinga
% förwarat.

When you want to Form first well, stand with whole or with hunched
Body, and extend the Blade well out forwards, at the height of your
Head, Blade and point straight to the end, as much as is
possible. Otherwise it is by nature to the Earth inclined, through
this the enemy cannot step in on the outside, where the blade is
weakest / must also be well covered. Have your lower body well bent,
feet together, and bow well, so that your Enemy cannot reach you
below, above your are defended\sidenote{This is an interpretation,
choosing to read 'förwara' (store, keep) as 'försvara' (defend,
protect), as that makes sense here and later} by your Blade.

% Och märk här i synnerheet eller i gemen / att du uti all guard' i de
% blott / som äro emot korset eller mot styrkan förer Klingan / och
% fördenskull med en ringa motion dig defendera kan; De samme föras
% fram / der emot de andre / som äro långt ifrån styrkian / äre
% blotte / tag dem wäl sinkos / der med de icke blifwa desz mindre
% förwarade.

And note here in particular or general, taht you in all guards i those
openings, that are against the hilt or the strong moving the Blade,
and thus ith a small movement you defend can. These are brought
forwards, against others, that are far from the strong, take them well
bent\sidenote{'sinkos' could be bent, it could also be slow}, through
this the will not be less defended.

% Denne gvardia är god emot hugg / emedan du utan handens förändring
% tillijka offendera och dig defendera kan / och wore fuller den bästa
% / om hon icke wore så swår och mödosam / af orsak att man intet
% länge kan opphålla sig der uti.

This guard is good against cuts, sine you can without changing your
hand offend, as well as defend, and would fully be best, it it wasn't
so hard and laborious, for reason that you cannot long stay in it.

% Secunda med rak Arm stöt / är och så god / och skall skee i gvard'
% med hela kroppen / altijd att wara des säkrare / bör man hafwa
% fötterne wäl tillsamman: Den samma är intet så mödosam som prima,
% emedan Armen der uti kommer något lägre; Du måste altijd söka att förhindra

Second with a straight Arm thrust, is also good, and should happen in
guard with the whole body, always to be safer, you should have your
feet well together. This is not as laborious as first, since the Arm
is somewhat lower. You must always seek to avoid