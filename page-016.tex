\newpage

\scan{016}

\newpage

% Translation below

% Men blifwo han stilla liggiandes, så fullfor du ditt string' på
% följande sätt. Står hans Klimga jämpt eller angulerat till hans
% högra sinkos / string' honom innan till; Men befinner han sig i de
% öfre secunda, tert' eller qvarta, med sin udd alldeles hgt så skeer
% dett med qvart; så att din kropp är innan till aldeles betäckt; Och
% giörs intet behof när Fienden går med sin Klinga utur presenza, att
% du förföllier henne / elliest torde letteligen komma med din hand in
% confusion.

But if he stays still, you can complete your bind in the following
way. If his blade is even to or angled towards his right, bind him on
the inside. But if he is in a high second, third or fourth, with the
point high, you should proceed in the fourth, so that your body is
covered on the inside. And do not make a need to follow your Enemy's
blade if it does out of presentation, otherwise you will easily get
your hand confused.

% Är hans udd intet högre än din högra Axel / string' du med halfwa
% qvarta; är han med udden emot din giördel / string med tertia och
% något förhängande udd; Men wore hans udd i ett medel läger något
% ophögt, så string' honom med ophöjder udd / allenast du för hans
% linie / som kommer utaf hans udd och går till din kropp äst
% fötsäkrat. Ligger han än lägre och i tert' / så stringera honom med
% forhängande tert' eller qvarta, innan till med qvarta och utan till
% med halfwa secunda;Men wore hans Klinga invertes wänder / eller
% dyt angulerat, så string honom utan till / ungefär med halfwa
% secunda eller med halfwa qvarta / emädan han med sin udd intet lägre
% ligger än ditt underliff / blifwer och med förhängiande tert'
% string'. Du må och utan till / så wäl emot de höga som under
% gvardien med hela qvarta stringera / särledes är du fächtar med [???
% förskränkter ???] kropp.

If his point is not higher than your right shoulder / bind with half
fourth; if his point is towards your girdle, bind in third amd
slightly depressed point. But if his point is in-between, slightly
high, bind with a slightly raised point, so that you move his line so
that you are safe from the line that comes from his point to your
body. If he is lower and in third, bind him with a hanging third or
fourth, inside with fourth, outside with half second. But if his blade
is turned inwards, or angled that way, bind him on the outside,
apprximately with half second or half fourth, while his point is no
lower than your genitals, you should remain with binding with a
hanging third. You may stay on the outside against both high and unde
the guard with teh whole fourth bind, especially when you fence with a
lowered body.

% Hämptar då Fienden desze mutationer flere gånger igen / måste du
% fördenskull intet blifwa otålig / utan den samma på nytt som
% tillförende med försichtigheet förföllia / till desz du får mensuram
% och tempo att stöta (Så kan och string' eller Contrapostur med [
% strånker ???] kropp förättas ) med särdeles nytta och fördehl emot
% din fiende adoprerat blifwa. En 1. när han Caverar. finner han de
% blott uppå dig vaa borttagne / och kan du i det samma jämpt gå in på
% honom 2. Ju mer din wenstra Axel kommer fram / ju starkare blifwer
% du med din Klinga / så att du kommer desz diupare i mens' och din
% kropp i dett samma utur faran. 3. Kam Fienden din klinga intet wäl
% passera 4. kan han intet så wäl som elliest bruka wenstre
% handen. 5. Går man på dett sättet säkrare emot de angulerte
% gvardien, emädan bägge blotterna / så wäl den invertes som utwertes
% blifwa der uti wäl defenderade. Men merck hos detta manier, att du
% dine tåår af hgre foten / i dett samma skänkningen skeer / wender
% utan till / så kan du mycket bättre skränka kroppen

Even If your Enemy uses these mutations multiple times, you must not
become impatient, but proceed with caution until you get measure and
tempo to thrust (likewise can binding or Cotrapostur with a strong
body be performed) with exquisite usefulness and advantage against
your Enemy be adopted. 1. when he Beats, he finds that your openings
are gone and you can at that time close on him 2. The firther your
left Shoulder is forward / the stronger you are with your blade / so
that you can go deeper into measire and your body out of danger 3. If
the enemy cannot pass your blade 5. You are in this way safer against
an angled guard, since both openings, the outward as well as the
inward are well guarded. But note that this manner taht you turn the
toes of your right foot outwards in the moment you lower your body so
taht you more easily can lower your body.

% Hwad Ligering anbelangar / är det intet annat än en
% Constrapostur. der utinann man med hele kroppen wäl bockar sig utan
% abbaissering af Armen . med förhängiande Klinga utan till Fiendens
% swaga / och hans udd derstädes förstängder; är der hos at merkia /
% det du emoot de nedre gvardien med hela secunda / och högre gvardien
% med halfwa secunda ligerar ; En om du hade stringerat och wille i
% dett samma stängia hans klinga utan till / så träd något mer till
% hans högra proportionalimente, utur förhängiande Tert', uthan till i
% den förhängiandeSecunda, så har du hans klinga der med ligerat. När
% du elliest ofwan uthföre wilt ligera, måste du med din Klinga rätt

As far as Ligering\sidenote{Both ``ligera'' and ``stringera''
translate to `bind'} is concerned / it is nothing but a Contrapostur,
with prior bending of the body without
bending\sidenote{``abaissering'', no actual translation found} the
arm, with a dangling Blade to th Enemy's weak / and thus his point
closed off; you should note that you bind with the whole second
against the lower guard and half second against the high guard. If you
have bound and want to close off his blade to the outside / step
slightly to his right, out of dangling Third, outside in the hanging
second and you have Ligerat his blade. When you otherwaise want to
Ligera on the outside, you must with your blade righly