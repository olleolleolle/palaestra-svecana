\newpage

\scan{028}

\newpage

% Translation below

% Der med du och i dina stöter / för den linien som kommer af hans udd
% till din kropp är desze bättre betäckt / han desz mindre parera kan
% / måste du i stöten med hela din styrkia i ans hela swaga indela;
% Men du måste intet med din udd gå så när till hans korsz /
% derigenom han intet med en lijten motion parera och i dett samma
% stöta / men wore hans rörelse så stor eller hans udd så långt ifrå
% dig / att du intet kan gå fort med hans Klinga / är dett bäst att du
% lägger dig i ett berett läger / och förer udden allena till den
% ohrten / hwar ifrån han kommer med sin Klinga / dett wari sig högt
% eller lågt / här med går man wäl betäckt in med stöten / och när
% du har stödt a pie fermo, effter fullbordad stöt geschwindt går
% till hans Klinga / der med du blifwer för hans Klinga betäckt; Och
% om linea recta intet som sombliga mena betecker / ty dett gifwes de
% under och öfwer blotten / och är Klingan intet så stor att hon
% fullkomligen Cooperera kan / så är då intet utan / ju mer man är i
% den recta linea, ju lättare och med mindre motion man parerar.

% Dett är och med ett särdeles Arcanum i Fechtning; En när du mens'
% largam occuperat hafwer / och lyffter opp foten till stöten / intet
% all för snar uti en furie infaller med stöten / men tillförende med
% utsträkter Arm och kroppen i stöten ett ögnebleck håller inne och
% stilla / hwilke ringa oppehald / dig nödwändigt ( på din
% Adversarius antingen i dett första tempo i hwilket han moverar ) går
% fort / eller i mensurens betagning när han intet rörer sig / genom
% hwilka medel / du kan alla din Fiendens parering till intet giöra /
% och honom mestdehls i dett första tempo råka.

% Emot de Auglerte läger tager du i acht / hwad som wijd stöterne
% tillförende är omtalt Cap 8. så märk här Generaluer {???], att du
% aldrig med eller utan någon stöt går fort när din Klinga är i
% string'/ du har henne då tillförende med eller utan retirade
% effter mensurens tillfälle frij giordt.

\chap{Thrusts a pie fermo}
