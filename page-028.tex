\newpage

\scan{028}

\newpage

% Translation below

% Der med du och i dina stöter / för den linien som kommer af hans udd
% till din kropp är desze bättre betäckt / han desz mindre parera kan
% / måste du i stöten med hela din styrkia i hans hela swaga indela;
% Men du måste intet med din udd gå så när till hans korsz /
% derigenom han intet med en lijten motion parera och i dett samma
% stöta / men wore hans rörelse så stor eller hans udd så långt ifrå
% dig / att du intet kan gå fort med hans Klinga / är dett bäst att du
% lägger dig i ett berett läger / och förer udden allena till den
% ohrten / hwar ifrån han kommer med sin Klinga / dett wari sig högt
% eller lågt / här med går man wäl betäckt in med stöten / och när
% du har stödt a pie fermo, effter fullbordad stöt geschwindt går
% till hans Klinga / der med du blifwer för hans Klinga betäckt; Och
% om linea recta intet som sombliga mena betecker / ty dett gifwes de
% under och öfwer blotten / och är Klingan intet så stor att hon
% fullkomligen Cooperera kan / så är då intet utan / ju mer man är i
% den recta linea, ju lättare och med mindre motion man parerar.

Through that in your thrusts you take the line that goes from his oint
to your body is better covered that he is less able to parry, you must
in your thrust ensure thta the whole of your strong goes into the
whole of his weak. But you must not with your point go so close to his
hilt that he can parry with but a small motion and thrust at the same
time., but were his motion large or his point far from you, taht you
cannot quickly reach his Blade, it is better that you go into a
prpared position, and brin solely the point to that location from
where he comes with his Blade, neither high nor low, through this we
go well covered in with the thrust and when you have thrusted a pie
fermo, after completed thrust, go promptly\sidenote{the original's
``geschwindt'' could be 'quick' or 'prompt'} to his Blde, through this
you become covered for his blade. And if linea recta doesn't, as some
say, cover, since it gives upper and lower openings, and if the Blade
is not so lareg that it can fully cooperate, it is not without that
the more you are in the recta linea, the easier and with less motion
you parry.

% Dett är och med ett särdeles Arcanum i Fechtning; En när du mens'
% largam occuperat hafwer / och lyffter opp foten till stöten / intet
% all för snar uti en furie infaller med stöten / men tillförende med
% utsträkter Arm och kroppen i stöten ett ögnebleck håller inne och
% stilla / hwilke ringa oppehald / dig nödwändigt ( på din
% Adversarius antingen i dett första tempo i hwilket han moverar ) går
% fort / eller i mensurens betagning när han intet rörer sig / genom
% hwilka medel / du kan alla din Fiendens parering till intet giöra /
% och honom mestdehls i dett första tempo råka.

% Emot de Auglerte läger tager du i acht / hwad som wijd stöterne
% tillförende är omtalt Cap 8. så märk här Generaluer {???], att du
% aldrig med eller utan någon stöt går fort när din Klinga är i
% string'/ du har henne då tillförende med eller utan retirade
% effter mensurens tillfälle frij giordt.

\chap{Thrusts a pie fermo}

% Hwar och en stx8"ot blifwer a pie fermo eller med en pasade
% anbracht. A pie fermo stöta / heter / när man blifwer med wänstra
% foten Ferm. och förer med den högre foten st"oten fram/ och går med
% dett samma ifrå sin Fiende tillbaka igen/ h/aller när man med stöten
% på sin Fiende utan foens rörelse / allena med kropsens öfwerböjning
% råkar; Desze brukar du när du är Arriverat i stricta. de andra när
% Fienden will Arrivera:

% De manier att stöta a. pie fermo är dett sembste sättet / derföre
% måste man först öfwa sig der uti der med man lärer braf  långt och
% stadigt stöta