\newpage

\scan{014}

\newpage
% transcription
% Ligger din Advers' med en jembn utsträckt Arm / och har wändt sin
% Klinga utan till, så gå du med din Klinga ifrå hans wenstra / till
% hans högra / och stringre honom innan til; har han wändt sin udd
% innan til; går du ifrå hans högra til hans wenstra / och stringera
% honom iutan till. Du kan och förr än du stringerar / gå med din udd
% i prospective hans kors / och så med den högre foten fram sättiandes
% iospeso stringera honom / om han då intet moverar sig / du med en
% hast då stoter i samma blott/ När du de under gvardien utan till
% will ligera / så låt din udd uti secunda jämpt under sig siunka /
% och eftter de sccheamater och mutationer de gvardia stringera / som
% altijd elliest med porterter Klinga / hwar med du i en hast som
% tillfälle gifwer / den ene effecten i den andre kan förwända; och
% sätt din foot intet för fort / förr än du har stringerat, dett ware
% då at man hade utag sin Wederparts rörelse / i dett man tillika med
% Klingan och Foten utur den wijda i den [Enga, Euga ??] Mensur avancera kan

% Translation below

If your adversary has an extended arm and has turned his blade
outwads, you should bring your blade from his left to his right and
[stringera] him on the inside. If he's turned his point inwards, you
enter from his right to his left and [stringera] him on the
outside. You can also before you [stringera] place your point [i
prospective] his [cross] and with your right foot moving forwards
[stringera] him, if he doesn't move as you thrust into the opening.

When you want to [ligera] on the oustide under the guard, let your
point drop in second and after schemes and changes you can [stringera]
the guard, as always otherwise with [porterer] blade with which you
with haste when an opportunity is given, can turn one effect into the
other; and don't move your foot too fast, until you have [stringerat],
this being that you can with Blade and Foot out of the wide into
[Enga] Measure can advance.

\chap{What Mensura Stricta is}

% Stricta, eller then Enga Mensur, är twäggiehanda / såsom Foten
% /Kroppen och Klingan. Mensura Stricta med Foten / är när du har födt
% foten itur den utur den wijda / och bracht honom i den enga Mensur,
% så at du med blotta öfverböjningen aff din Kropp kan tillräcka din
% Fiende.
% Mensura Stricta med Kroppen / är utan fortsättning din foot /
% allenast med Kroppens öfwersänkning uthur Mens. larga i
% mens. strictam
% Mensura Stricta med Klingan är den distans dijt du med din Klinga
% igenom kropsens öfwerböjning utur den wijda i den ENga Mensur
% bringar.
%
% När du nu denna ensur hafwer bekommit (hwilket med stor
% försichtigheet och porterer Klinga motte förättas) kan du på alla
% din Advers Motioner stöta; och der han intet skulle movera sicg /
% ändock igenom den fördel med Contrapsotur aff din [styrkia???] /
% instöta i hans svaga; Men der han skulle uti sin gvardia wara
% temmeligen Copert / år du ändå med din stöt och deportere klinga
% fort. Men om han skull parera / muterar du din effect / och rätar i
% den andre ypningen som kom utaff hans parering.
