\newpage

\scan{005}

\newpage

% Translation below
\chaptitle{To the inclined reader}

Among all the things under Heaven, we hear in the all-holy Words of
God who are revealed in the 5th Book of Moses,\sidenote{Probably} 20th chapter,
verse 8, how the captains of the Jews have spoken to the soldiers and
those who are afraid and keep fear in their hearts should stay behind
so that they do not place fear in the herats of their brohers. In
[lifta] motto, the pugnacious hero Judas Macabeus (who was brave as a
lion and unafraid like a young lion on the prowl, for this he was
highly regarded in Israel and among all heathens) shouted to his
soldiers that those who were afraid should go back home. Wherein we
see the all-highest's curious dislike for man's womanly and timid
heart.

As they are forbidden that they were not trusted to see to their own
defence and that of the Fatherland. As they're incapable of (to such a
needed and manly endeavour) but with their fearful mind among their
friends and followers. 

Though these means are not only most falls and defeats accomplished
but many a city and country are so found in other hands, with
self-reliance thrown away.

For this sake all famous Captains in the past wold (I want to
keep silent about the now) have been eager to have not many, but good
and capabe soldiers, which history carefully shows (like a Mother of
[alltidfhet?]) how a small troop of brave and heartful Warriors
frequently have both beaten and overthrown a large army and thus
conquered land and Countries.

That fewer people can sometimes accomplish more than a multitude,
among other high [Actioner] can be seen with Alexandrum Magnum who
with his well-trained Macedonians (who was nonetheless a small power)
vanquished the whole Oriental world and its highest Regiment.

Even though there were Numerus Imbellis non Miles, why do the old
times' regents always have taken the trouble to cleanse the
useless and so as soon as they have sensed fear in a soldier they have
fired\sidenote{In the sense of terminating employment..} and
separated him from the army. 

Just as Annibal, when we was about to
commence the war against the Romans  he was forced to send home 7000
Spaniards\sidenote{This is a guess from [Spanier]} in which he sensed
fear and low morale, who would be more harm than help his
War-horde. Why as they in all times have coveted milites eruditos, so
have always in the forn times well-served regiments their own most
noble problem been that within every kind of Virtuous Exercitier their
youth and manpower [ovtuchta] and [målskicka].

And who would well imagine me with a well-armoured Soldier such
completeness that he is a Person having received of unusual powers
from Nature and a strong and stable body to be able to withstand all
sorts of military work and umbrage, battling hunger and misfortune, to
sustain heat, water and cold.
