\newpage

\scan{025}

\newpage

% Translation below
% Fiende att parera och hwar till han är benägen hämpta utur honom;
% Men man måtte taga sig hastig tillbaka igen / och med den förste
% Finten intet gå diupt som elliest / man kommer sig medelst i den
% andra så diupt in / att man gifwer sig i fara / särdeles när man
% ingen rörelse af sin fiende har för sig.

Enemy to parry and to which he is apt to [bring out of
him???]\sidenote{the Swedish is hämpta utur honom, that is a literal
translation}. But one must recover quickly, and not go  deep with the
first Feint, as one would on the second end up so deep in, that one is
in danger, especially when there's no movement from one's enemy.

% Och effter man tillförende om den fördehl so Foten hafwer /
% sospeso in aria någorlunda handlat / will iag dett något tydligare
% specificera och förklara. Till dett första/ är den oundwijkeligen
% att den samma som i stricta mens' sin foot i string' och finter
% sospesi in aria har / och imedlertijd hwilar kroppen på den wänstre
% foten / der med han sin Adversarij motion Effecten ändra kan; 
And after one has brought the advantage gained from suspending the
foot in the air somewhat handled / I want to somewhat clearer specify
and explain. To the first/ it is unavoidably that the same who in strict
measure has his foot in a bind and feints hanging in the air / and
meanwhile rests the body on the left foot / through that can change
his Effect based on his Adversary's motion.
%   En när
% den andra har satt sin foot nehr och will lyffta honom opp till
% stöten/ har den andra/ som hade sin foo ifrå Jorden/ satt honom i
% dett samma neder/ och genom Vantagio af Constraposttur i forste
% tempo rkat. I lijka måtto der din Fiende skulle liggia stilla när
% du har string' honom sospeso in aria:
When the other has planted his foot and wants to lift it to the
thrust/ the other, who has his foot off the Earth, placed it down, and
through Advantage of Contraposition, in the first tempo thrusted. In
similar measure, your Enemy should stay steady when you have bound him
suspended in the air.
%    En när du har bracht foten i
% strictam, går han genom en lijten Motion geschwindt fort/ när han
% effter giord stött A. p. formo går tillbaka igen med string' och
% achtar sig för effter stöterne/ behållandes den fremste foten
% sospeso in aria , kan han med den föhrdel af Contrapostur, eller i
% dett första movement och tempo återstöta;
When you have brought your foot into close measure, he goes with a
small motion very quickly/ after which he will thrust [A
. p. formo???] recovers with a bind and is wary of afterthrusts,
keeping the front foot suspended in the air, he can with advantage
from Contraposition, or in the first movement and tempo re-thrust.
%   En om han har fått foten
% nehr/ har stöten kommit mycket långsammare/ än när man (i
% parerinegn) håller foten sospeso in aria, kan man mycket förr gee
% Risposten, än om man hade satt honom nehr/ när man har foten i
% Contrapostur, eller i string' sospeso in aria, kan man uti alla
% finter, der din Contrapart sin fot har satt nehr, säkert att stöta;
When he has planted his foot / the thrust will come much slower / than
if he (in the parry) holds the foot suspended in the air, can one much
quicker give Riposte, than if one has put the foot down. When you have
your foot in Contraposition, or in a bind suspended in the air, one
can in all feints, where your Counterpart has put his fot down, thrust safely:
% En medan han satte foten nehr/ när han giorde finten, har han intet
% kunna taga Contratempo, der emot den andra so hade sin foot i finter
% sospeso in aria ( när Fienden i den samma har welat stöta) wäl
% kunnat giordt/ måste du din fint i dett samma din Adversarius
% parerar i en stöt blifwa förändrat/ hwilker den samme som i finten
% har satt foten neder / a Tempo intet har kunnat giordt; Men man
% måtte i alt detta intet lyfta foten högt/ utan att han när wijd
% Jorden blifwer oppehållen.
And while he put his foot down / when he did the feint, he has not
been able to take Countertempo, against the other who had his foot in
the feint suspended in the air (when the Enemy in the same wanted to
thrust) well been able to do, you must change your Feint in the same moment
your Adversary parries a thrust, which the one who in the feint put his
foot down will not be able to do in Tempo. But one must not in all
this lift the foot high, but just hold it off the Earth.

% \chap{hwad som Offosa i stöten är i gemen}

\chap{What is [often???] in the thrust common}

% Dett är när somblige de willia stöta / draga de tillförende Armen
% åht sig och i dett samma skiuta de med gewalt åter irfån sig /
% wråkandes eller kanstandes i mening / att så derigenom en större
% efftertryck. Men ett sådant manier, är effter fölliande orsaker
% intet godt eller gagneligit. Först / när Fienden täncker att råka
% de blott / kan han för sådan kastning af Klingan / de motioner som
% äro aff nöden/ intet förändra: Och der man kulle taga swagan i acht
% / bringar man udden eller dett swaga förr utur presensia och

Some are in the thrust prone to pulling their arm towards them, then
shoot it back forwards, wreaking or throwing in some sense, to give a
larger impression. But such a manner is for the following reasons not
good or gainful. Firstly, when the Enemy thinks to thrust into the
opening, he can in such a throw, the moves that are needed, not
change. And where one should mind the weak / it brings the point or
the weak out of presence and
