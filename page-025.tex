\newpage

\scan{025}

\newpage

% Translation below
% Fiende att parera och hwar till han är benägen hämpta utur honom;
% Men man måtte taga sig hastig tillbaka igen / och med den förste
% Finten intet gå diupt som elliest / man kommer sig medelst i den
% andra så diupt in / att man gifwer sig i fara / särdeles när man
% ingen rörelse af sin fiende har för sig.

Enemy to parry and to which he is apt to [bring out of
him???]\sidenote{the Swedish is hämpta utur honom, that is a literal
translation}. But one must recover quickly, and not go  deep with the
first Feint, as one would on the second end up so deep in, that one is
in danger, especially when there's no movement from one's enemy.

% Och effter man tillförende om den fördehl so Foten hafwer /
% sospeso in aria någorlunda handlat / will iag dett något tydligare
% specificera och förklara. Till dett första/ är den oundwijkeligen
% att den samma som i stricta mens' sin foot i string' och finter
% sospesi in aria har / och imedlertijd hwilar kroppen på den wänstre
% foten / der med han sin Adversarij motion Effecten ändra kan; En när
% den andra har satt sin foot nehr och will lyffta honom opp till
% stöten/ har den andra/ som hade sin foo ifrå Jorden/ satt honom i
% dett samma neder/ och genom Vantagio af Constraposttur i forste
% tempo rkat. I lijka måtto der din Fiende skulle liggia stilla när
% du har string' honom sospeso in aria: En när du har bracht foten i
% strictam, går han genom en lijten Motion geschwindt fort/ när han
% effter giord stött A. p. formo går tillbaka igen med string' och
% achtar sig för effter stöterne/ behållandes den fremste foten
% sospeso in aria , kan han med den föhrdel af Contrapostur, eller i
% dett första movement och tempo återstöta; En om han har fått foten
% nehr/ har stöten kommit mycket långsammare/ än när man (i
% parerinegn) håller foten sospeso in aria, kan man mycket förr gee
% Risposten, än om man hade satt honom nehr/ när man har foten i
% Contrapostur, eller i string' sospeso in aria, kan man uti alla
% finter, der din Contrapart sin fot har satt nehr, säkert att stöta;
% En medan han satte foten nehr/ när han giorde finten, har han intet
% kunna taga Contratempo, der emot den andra so hade sin foot i finter
% sospeso in aria ( när Fienden i den samma har welat stöta) wäl
% kunnat giordt/ måste du din fint i dett samma din Adversarius
% parerar i en stöt blifwa förändrat/ hwilker den samme som i finteb
% har satt foten neder / a Tempo intet har kunnat giordt; Men man
% måtte i alt detta intet lyfta foten högt/ utan att han när wijd
% Jorden blifwer oppehållen.

% \chap{hwad som Offosa i stöten är i gemen}

% Dett är när somblige de willia stöta / draga de tillförende Armen
% åht sig och i dett samma skiuta de med gewalt åter irfån sig /
% wråkandes eller kanstandes i mening / att så derigenom en större
% efftertryck. Men ett sådant manier, är effter fölliande orsaker
% intet godt eller gagneligit. F"orst / när Fienden täncker att råka
% de blott / kan han för sådan kastning af Klingan / de motioner som
% äro aff nöden/ intet förändra: Och der man kulle taga swagan i acht
% / bringar man udden eller dett swaga förr utur presensia och