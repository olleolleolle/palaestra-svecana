\newpage

\scan{034}

\newpage

% Translation below
% uti ett tempo tilijka stöta kan / så framt han icke brukar en sådan
% Subtilitet, att han råkar den andre / förr än han råkar eller
% berörer hans Klinga / elliest är dett bättre att han undflyr med
% Klingan hans stöt / och i dett samma befrijar sin Klinga / måste en
% heller alt för mycket nalkas in till / utan alle sine defensioner
% med en lijten retirade af kroppen / der med att stöten icke blifwer
% honom så swår att parera, men uti andre tillfällen skall man hålla
% udden emot sin Fiende / der med han intet kan passera.

in one tempo also can thrust / unless he uses such a Subtlety, that he
reaches the other / before he reaches or touches his Blade / otherwise
it is better to flee the thrust with the Sword / and in that moment
free one's Sword / must also not get too near / but all defences
should be done with a slight retirement of the body / through this
making the thrust easier to parry, but at other occasions one should
keep the point towards one's Enemy, thus he is unable to pass.

% Är fördenskull intet rådsamt / att en Swager mot en Starkare söker
% med stringerande taga mensuram, utan skall hålla sig långt ifrån /
% och icke låta finna sin Klinga / men offta gifwa ypningar eller
% elliest någon occasion och tempo till stöten provocera, eller och
% ibland tillbinda honom Klingan / dett han kan tycka att han hafwer
% henne / och när han i dett samma wille röra sig / i tankar att
% passera, kan man bryta mens' och (i de blott som han gifwer sig /
% eller för än han sig retirerar) råka / och begifwa sig med en hast
% tillbakas igen / att han Salverar sig för hans stöt / och på sådant
% sätt med fördehl af distancen sig för honom defendera, och beskyddat
% blifwa.

It is not advisable / that a Weak against a Stronger seek to take
measure with a bind, but should keep distance / and not let his Blade
be found / but often resent openings or at least give occasion and
tempo to provoke a thrust, or sometimes
[tillbinda???]\sidenote{tillbinda would mean approximately 'bind
unto'} him the Blade / so that he thinks he may have it / and in the
moment wants to move / thinking to pass, you can break measure and (in
the opening he gives / or before he retreats) reach / and quickly
retreat again / that he Saves himself for his thrust / and in this
manner with advantage of distance against him defended, and protected is.

% Alltså när du hafwer att giöra med en Cholerisk eller med en wredsam
% / måtte man än mer retirera och till stöten reta / gifwandes honom
% bedrägeliga blotten eller falska tempo, der med han desz hastigare
% stöter / och i dett samma han dett giör / låter löpa honom på udden
% / eller effter lägenheten / öfwer hans ingång lijtet retirera, att
% du desz fullkomnare för honom åft defenderat, och han i dett samma
% tempo (för än han har passerat) blifwer läderat; En låta honom
% ostötter komma till ringning eller grijpa ditt fäste anlär intet
% vara rådsamt eller gagneligit.

So when you have to do with a Choleric or an angry / you must even
more retreat and provoke a thrust / giving him deceptive openings or
false tempo, thus he will thrust more quickly / and at the same time
he does / let him run onto the point / or based on circumstances /
over his entry slightly retreat, that you are more fully defended
against him, and he in that tempo (before he has passed) becomes
injured. To allow him unthrusted to come to
[ringning???]\sidenote{This word has occured before...} or to grip
your hilt would not be advisable or to your advantage.

% Anbelangande en Phlegm, som sig länge betänker och med angripningen
% föredröjer / skall du på stunden asselera; Men altijd med got
% förnuft och upsicht der med att du intet sielf blifwer bedragen; En
% dett sker offta / att en (igenom den begiärelse och längtan som han
% drager att stöta sin Fiende) blifwer der wijd råkat och läderat, ty
% om han har gått in med god försichtighet / hade han lätteligen
% kunnat defendera sig / och i dett samma stöta den andre.

As concerns a Phlegmatic, who considers deeply and delays the attack /
you should immediately assail. But always with good sense and
vigilance so that you do not become fooled. It often happens / that
one (through the desire and longing that he draws to thrust his Enemy)
become through this reached and injured, since if he had gone in
carefully / he could have easily defended himself / and in that moment
thrust the other one.

% Är fördenskull i acht tagandes / man hade nu att giöra med hwem man
% wille / att man dett Confiderar, icke alt för ringa Estimerar sin
% Fiende och förachtar honom / utan mycket mera bruka försichtigheet på
% alt hwad sig kan tilldraga / der med man uti alle tilfällen promter
% och expediter sig befinna må.

It is nonetheless to note / that if you have to do with whomever you
wanted / that in Confidence, not too little Estimate one's Enemy and
despises him / but much more use care in all that can happen / thus
you in all occasions will find yourself prompt and expedient.

% Men när man finner hoos en Fiende / meer en infödd redhoga än
% argliftigheet

But when you find in an Enemy / more a bred-in fear\sidenote{This is
assuming that 'redhoga' is a cognate of ``räddhåga''} than anger
