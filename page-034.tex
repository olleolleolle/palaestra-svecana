\newpage

\scan{034}

\newpage

% Translation below
% ut ett tempo tilijka stöta kan / så framt han incke brukar en sådan
% Subtilitet, att han råkar den andre / förr än han råkar eller
% berörer hans Klinga / elliest är dett bättre att han undflyr med
% Klingan hans stöt / och i dett samma befrijar sin Klinga / måste en
% heller alt för mycket nalkas in till / utan alle sine defensioner
% med en lijten retirade af kroppen / der med att stöten icke blifwer
% honom så swår att parera, men uti andre tillfällen skall man hålla
% udden emot sin Fiende / der med han intet kan passera.

% Är fördenskull intet rådsamt / att en Swager mot en Starkare söker
% med stringerande taga mensuram, utan skall hålla sig långt ifrån /
% och icke låta finna sin Klinga / men offta gifwa ypningar eller
% elliest någon occasion och tempo till stöten provocera, eller och
% ibland tillbinda honom Klingan / dett han kn tycka att han hafwer
% henne / och när han i dett samma wille röra sig / i tanar att
% passera, kan man bryta mens' och (i de blott som han gifwer sig /
% eller för än han sig retirerar) råka / och begifwa sig med en hast
% tillbakas igen / att han Salverar sig för hans stöt / och på sådant
% sätt med fördehl af distancen sig för honom defendera, och beskyddat
% blifwa.

% Alltså när du hafwer att giöra med en Cholerisk eller med en wredsam
% / måtte man än mer retirera och till stöten reta / gifawandes honom
% bedrägeliga blotten eller falska tempo, der med han desz hastigare
% stöter / och i dett samma han dett giör / låter löpa honom på udden
% / eller effter lägenheten / öfwer hans ingång lijtet retirera, att
% du desz fullkomnare för honom åft defenderat, och han i dett samma
% tempo (för än han har passerat) blifwer läderat; En låta honom
% ostötter komma till ringning eller grijpa ditt fäste anlär intet
% vara rådsamt eller gagneligit.

% Anbelangand en Phlegm, som sig länge betänker och med angripningen
% föredröjer / skal ldu på sinnden asselera; Men altijd med got
% förnuft och upsicht der med att du intet sielf blifwer bedragen; En
% dtt sker offta / att en (igenom den begiärelse och längtan som han
% drager att stöta sin Fiende) blifwer der wijd råkat och läderat, ty
% om han har gått in med god försichtighet / hade han lätteligen
% kunnat defendera sig / och i dett samma stöta den andre.

% Är fördenskull i acht tagandes / man hade nu att giöra med hwem man
% wille / att man dett Confiderar, icke alt för ringa Estimerar sin
% Fiende och förahtar honom / utan mycket mea bruka försichtigheet på
% alt hawd sig kan tilldraga / der med man uti alle tilfällen prmoter
% och expediter sig befinna må.

% Men när man finner hoos en Fiende / meer en infödd redhoga än argliftigheet