\newpage

\scan{051}

\newpage
\sidenote{The first half of the first paragraph of this page is most of the
last paragraph from the last page, translation continues from where
the previous page ends.}
% Translation below
% andra går iempt för sig och råkar; Der emot passerar den ene angulum
% genom den andre / och råkar utan resistance af den andre; ja som
% mera är / Cederar den ene den andre / mycket snarare råka de / och på
% sådant sätt passera de emot de jämbde linier och träffa således
% wijd Fiendens rörelse; [HERE] Så komma och de angulerta stöter
% långsammare / och hinnan intent så långt / som de hwilka blifwa
% giorde i en jämbd linie: In summa de angulerte stöter råka wäl. Men
% till defension icke så gagnelige / och den som wil lgå säkert
% emot dem / måtte icke allenast förstå Klingans avantage, utan der
% jämpte wetta wäl att bruka kroppen och fötterne / såsom och första
% Finten in angulo wäl / elliest blifwer han tilliika råkat.
However, the angled thrusts are slower, and cannot get as far, as
those who are made in a straight line. To sum up, the angled thrusts
reach well. But they are less useful for defence, and he who wants to
go safely against them, must not only understand the advantage of the
Blade, but also know well to use the body and the feet, like the first
Feint in angle well, otherwise he will be reached.

% Dett tridie manieret är bättre / der Armen blifwer något retirerat,
% och Klingan har lijka såsom en justam lineam ifrån Amrbågan till
% udden pa sådant sätt kan du bättre få avantage att stöta och parera,
% och när det giörs behoff Med mehr färdighet cavera; Kroppen är mera
% aff Klingans styrkia betäckter/ och behåller udden bättre i
% presenza; imedlertijd är dett gagneligt att man weet snart det ene
% som det andre manieret i fall at bruka/ i dett att hwar han intet
% har practicerat; kan han den andras natur / och hwad som äf honom
% herröres / intet så lätt ehrkänna. Så måste man en heller tänckia
% / ett en ahrt tienar emor alt / utan hwar och en under sig hafwer
% sin egen terminum och måhl / och hwad som emot ett är godt / må
% intet brukas emor ett annat: Altså att man är försedder med månge
% lectioner, och huru de samme a tempo blifwa anbrachte / måtte wara
% instruerat.
The third manner is better, where the Arm is somewhat retracted, and
the Blade has an even line from the Elbow to the point. In such a
manner you can better get advantage to thrust and parry, and when it
is needed, with more skill disengage. The body is covered better by
the Blade's strong, and keeps the point better in presence. It is
however advantageous to know how to use the first as well as the
second maner, in that what he has not practiced , he can the nature of
the other, and what will him [herr\"{o}res\sidenote{Huh? No idea what
this means...}], not so easily acknowledge. And one must not think
that one art serves against all, but each and everyone has as an
underpinning its own end and goal, and what is good against one,
should not be used against another. THus one must be provided with
many lessons, and how these in tempo can be brought on, must be
instructed.

% Men der du skulle dig än bättre med din kropp och Klinga situera /
% måste du intet föhra din Arm alt för mycket sträckt / dock något mer
% uthsträckter än till dig dragen / och med Klingan i en recta linea,
% eller något innan eller utan till effter Fiendens gvard', så är
% kroppen så mycket säkrare och blifwer rörelsen mindre [???
% känder???]; Och der Armen intet blifwer alt för mycket sträckter /
% warder Klingan der igenom fast, mehr stärkter och krafftigare;
% Desze gvard äro intet så mödossamme / utan starkare emädan Fienden
% intet så lätt kan då passera under klingan sinkos, utan rålar i
% stöten mycket längre in / kunna och Finterna beqwämare så medelst
% utan stort tempo med foten blifwa giorde / Så är man och klingan
% mechtigare / u dett att man pa mångehanda sätt kan arbeta med henne
% / och sig effter tillfälle på åthskillige manier situera: Och der
% man då förer styrkan på den ort som sig bör/ är man för sin Fiende
% säker och betäckter; Altså att denna gvardien, ( ibland de
% förbemelte att formera emot sinn Fiende) är den bästa. Dett är
% fuller det rådsammeste / at man sig intet låter uti någon gvard'
% emit sin Fiende; En wore den ene säkrare än den andra / hafwa de
% ändock alla em defect och feel / och kan fördenskull en förståndif
% och wäl öfwat fächtare (emedan i denna konsten den blotta Science, utan
% practica intet gäller) När han seer sin Wederpart uti ett läger
% Ferm, hwar utaf han honom icke allenast ehrkänner / utan weet i dett
% samma huru han skall gripa honom an ocg stöta; Kan och der utur
% afftaga hwad hans Fiende så wäl i offesa som diffesa förmår och
% hafwer wettenskap.