\newpage

\scan{047}

\newpage

% Translation below

% dett din Fiende intet binder dig utan till / fördenskull håller du
% ända fram för dig udden och högre Axeln lijft i en jämbd linea,
% bättre lijtet för högt / än att man skulle försänka henne; En der
% har hon lijka egenskap som prima, att hon der som swagast är; Och
% kunde din fiende lätteligen få dig till att parera, och derpå
% passera unde din Klingar bort: Der han skulle utan till Artaqvera
% [attaqvera???], måste du cavera i secunda, dock utan Approchering
% eller annalkande/ dett wari sig / att du i dett samma han nalkas /
% att du kunde stöta de inwärtes stöter / kan du med qvarta lätteligen
% defendera dig . af orsak att Klingan blifwe mycket längre
% uthsträckt. Secunda är och fördelachtigt nog / sig der uti att
% gvardera, när underdelen af kroppen blifwer wäl borttagen / at
% Fienden intet kan räcka honom . med mindre han icke tillförende
% stänger hans Klinga / hwilket ändå blifwer swårt att giöra / emedan
% man med ringa motion och möda der uti kan Cavera; Men dett är och
% swårt att länge oppehålla sig der uti.

that your Enemy doesn't bind on the outside / for this you hold the
point well in front of you and lift your right shoulder in an even
line, better a bit too high than one would lower it. This gives
similar properties as first, that it is weakest there.\sidenote{Go
back and check this through? this looks like a continuity problem} And
were your enemy easily to get you to parry, and thereafter pass inder
your Blade away; Should he attack\sidenote{As far as I can make out
the original, the text uses ``Artaqvera'', which as far as I can tell
means nothing, but if it reads ``Attaqvera'', thics could be taken as
an odd spelling of ``attackera'', the base form of 'to attack'} on the
outside, you must disengage in second, although without Approaching /
it being that, that you in the moment he closes, that you can thrust
on the inside / you can easily defend in fourth, because the Blade
becomes much further out-stretched. Second is advantageous enough,
though / since in this guard, if the lower half of the body is well
kept away / tha tthe Enemy cannot reach it, with less than he closes
his Blade / which is still difficult to do, since you can with small
motion and effort Disengage. But it is difficult to hold this position
for long.

% Öfvertertia wäl sträcketer och med hela kroppen giord / är den
% bästa gvardia sin Fiendes Klinga der uti söka och stringera / emädan
% man snart der utur kan gå i qvarta eller secunda, effter som tijd
% och lägenhet sig gifwer.

Overthird well extended and with the whole body made, is the best
guard to seek and bind one's Opponent's blade, since one quickly can
go to fourth or second, as time and opportunity is found.

% När då i denna tert' din Klinga på något sätt blefwa occuperat,
% måtte du med udden i Undertertia per lineam obliqvam gå till Jorden
% / och din kropp och knää som tillförende woro öfwerbögde / måtte
% aldeles retireras så åft du sådan fara frij / och hafwer betagit
% honom den mensur' till att stöta; Och om han då skulle på nytt söka
% att string' din Klinga eller inträda/ kan du honom i det samma
% allena med kropsens öfwerböjning / förutan att röra foten stöta
% och han intet så lätt få din Klinga / emädan udden står åth
% marken / och om han intet weet at bruka den vantagio med sin krop /
% i dett samma han föllier henne effter / blifwer han säkert stötter
When in this third your Blade in any way is occupied, you must with
the point in Underthird in an oblique line\sidenote{``per lineam
obliqvam''} to to Earth, and your body and knee were already bent /
must always retreat so you are free of such danger, and have taken
from him measure to thrust. And if he were again to seke to bind your
Blade or step in, you can in that moment alone with a bending of the
body, without moving your foot thrust and he will not so easily catch
your Blade, since the point is towards the ground, and if he does not
know to use the advantage with his body, at the same time he follows
it, he will surely be hit.
% Af orsak de distancier uti förberörde gvardia äro så bedräglige/
% att en menar dett han är långt ifrå sin Fiende / då den andre med
% blotta kropsens öfwerböjning/ mehr än med halfwa Klingan / utan att
% röra foten sig så mycket utur men' tillbakas begifwa kan / och så
% framt den andre icke har gvard' natur förstådt / lär han komma
% honom längre in än han tänkt hafwer: Em så mycket som han gifwer
% sig tillbaka / kan han och gifwa sig fram igen / och med offesa och
% diffesa tillijka gå in på honom / så att hans Klinga intet blifwa
% stringerat / utan icke Fienden kommer honom der med i den enga
% mensura, om han hölle då sine fötter nähr tillhopa och bögde
% kroppen så mycket öfwer som möijeligit wore / så lähr han hinna
% hans udd.
