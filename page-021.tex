\newpage

\scan{021}

\newpage

% Translation below
% derföre intet kan komma med din klinga i presens; Derföre giör du en
% mezza cavation, af orsaak / att den samma hastigare blifwer
% effectuerat, och in??? [ inskter ??? ] med det samma kroppen utur
% presens hans udd. Denna mezza cavtion blifwer intet altijd uti
% förste tempo utan mestendehls i den andre eller tridie motion, hwar
% effter mens' och stringerande blifwer när eller fierran tagen/
% practicerat och anbracht.

thus cannot bring your blade into presence; Therefore you do a mezza
cavation, because that is is quicker to be effectuated, and moves the
body out of presence of his point. This mezza cavation is not always
in the first tempo, but mostly in the second or third move, after
which measure and binding will be closely or distantly taken,
bracticed and brought on.

% Gynnes fördenskull / som ofwan är omtalt/ att man med alla
% cavationer ( de som sker a tempo, om man fulfuliförer sin effect )
% råka kan / eller man kommer sa långt fram/ att man i den
% nestfölliande rörelsen träffa kan; hwilket är dett rätta manier att
% operera;

Favour for this, as is above written, that you with all disengages (those
who happen in tempo, if you complete their effect) can reach, or get
so far forward, that one in the next movement can hit; which is the
right manner in which to operate.

% Men blifwa cavationer giorde utan tempo, måtte man offta/ emedan
% mens' är så när/ med en retirade cavera, der men man i medlertijd
% caverar, af Fienden intet blifwer råkat.

But ifdisengages are made outside tempo, one must frequently, since
measure is so close, disengage with a retreat, where one disengages between
time, to not be reached by the opponent.

% Derföre så offta din Fiende är in larga, eller i nermare mens' med
% en inträdning/ och föllier din swaga/ måste du cavera, ty uti det
% swaga är ingen force eller krafft att stringera eller
% parera. Caverar din contrapart utan retirering, när du redan har
% wunnit med din styrkia hans swaga/ giörs intet behf att du caverar/
% utan går i en jembd linea ända fram/ så att du far derigenom ett
% tillfälle på en annan sijda att lædera/ emädan han går med sin swaga
% i din styrkia/ och blifwer hans Tempo mucket större som igenomgår /
% än den som går ända fram; En den som caverar måtte giöra twå Tempo
% ett i cavaden, dett andr i stöten; Der emot den som stöter giör en
% motion i dett han går ända fram / derföre den som will wäl cavera/
% måtte intet gå alt för högt med sin udd/ utan något försänkt / så
% blifwer din contrapart intet så snart din cavation warse/ och när
% cavation är ute/ måste stöten wara ändat. cavera skeer på twänne
% sätt; först de som hålla uht/ caverar man med en Circel/ så får man
% sin Fiendes swaga des bättre/ och är wäl för säkrat för sin Fiendes
% uhthållande och stöt; Men der din Fiende är i motu, caverar man med
% en oval/ och lärer din Fiende beswärlogen kunna parera någon stöt/
% han må wara så hastig som han will/ dock röres intet mera än främsta
% leden der man hafwer Werjan uti.

Thus as your Enemy is frequently in wide, or closer with a step in,
measure and is following your weak, you must disengage, since there is no
force in the weak to bind or parry. If your couterpart disengages without
retiring, once you have won with your strong his weak, there is no
need to disengage, but just move in a smooth line all the way, so as to
giev you an opportunity to harm on the other side, since he is placing
his weak in your strong and his tempo is larger, as he going through
rather than just ahed. He who disengages need to take two tempos, one in
thedisengage and one in the thrust. Contrasted, he who thrusts only does
one motion, in that he goes the whole way. Thus he who wants to disengage
well must not place his point too high, but should leave it slightly
lowered, and that way your counterpart does not become aware of your
beat, and once the beat is out, the thrust must be finished. Disengaging
is done in two ways; first those who hold out, you beat with a Circle,
so as to catch your Enemy's weak better and stay well-protected
against your opponent's hold and thrust; But when your Enemy is in
motion, disengage with an oval, and your Enemy will have a hard time
parrying any thrust. He may be as quick as he wishes, but you are not
moving more than the foremost joint you hold your Rapier in.

\chap{What Chiamater are, and how they are used}

% Chiamater eller de bedräglige ypningar och blott/ är i dett
% 5. Cap. under contratempo påmint/ huruledes man skall gå emot dem/
% föllier altså hár huru man och säkert A Tempo kan bruka dem.

Chiamater or the deceitful openings, it is reminded in the 5 Chap. under
contratempo, how to go against them, it follows here how you safely
can use them A Tempo.

% Chiamater äro de blott / sedan man genom fördel af contrappostur /
% eller med en linea recta har fåd den Enga mensuram / att man d/a går
% med sin udd / tager sin Fiendes Klinga / och gifwer på sig en falsk
% ypning/ så att man låckar sin Fiende till att stöta.

Chiamater are the openings, once you have advantage through contrapostur
or with a linea recta have achieved narrow measure, that you then go
with your point, take your Enemy's Blade, and give a false opening, so
as to tempt your Opponent to thrust.