\newpage

\scan{026}

\newpage

% Translation below
%
% utan något motstånd / dijt der han elliest af naturen falla måste /
% altså förnär på detta sätt mer swagan / för den som parerar, än den
% andras styrkia som stötte; Derföre går sådanna stöter mycket owist /
% och hinna intet den ort der man egenteligen tänkt anbringa dem.

without any resistance / towards where he otherwise by nature must
fall / thus in this manner more the weak for he who parries, rather
than the strong of he who thrusts; Therefore thursts like these go
unwise / and cannot reach the place where one had intended them to reach.

% Öfwer detta så kan ingen till stötens ändemål hwarken Arm eller
% Klinga / hålla så stadigt att han icke siunker och gifwer altså
% Fienden ett stort tillfälle till stöten; Der till med när man en
% gång har sludrat eller wräkt stöten/ är man andre gången intet
% mechtig / utan man icke måste draga Armen tillbaka / hwilket är
% ett stort Tempo; En så framt Fienden i den förste Caduren intet har
% stött / har han då kunnat giordt när han drog Armen tillbaka / och
% för an sländge med Klingan retirera sig tillbaa / och åter igen med
% twå tempo parera och stöta kan. Dansedt de samma som på dett sättet
% kasta Klingan inge duglige Finter giöra kunna/ endels de röra foten
% eller kroppen utan Avancement af Klingan / men om de ändå skulle
% Avancera, dra de henne ofta så hårdt tillbaka / menandes der igenom
% giöra stöten dess hårdare / hwilket är ett långsamt och skadeligit
% tempo.
Furthermore no one can, as the thrust is concerned, hold Arm or Blade
so steady that he doesn't sink and thus give the Enemy an opportunity
to thrust. Additionally once you have fumbled or wrecked your thrust,
you are the second time not powerful, but you must pull your arm back,
which is a large Tempo; As far as the Enemy has not thrust in the
first [Cadur ???], he has then been able to when he pulled his arm
back, before throwing the blade forwards retire backwards, and again
in two tempos can parry and thrust. Dancing in the way of those who
throw the Blade means no useful Feints can be made, since they move
the foot or the body without Advancing the Blade, but if they were to
Advance, they frequently pull it back too hard, intending to through
that make their thrust that much firmer, which is a slow and harmful tempo.


% Andra äro åter / när de kan hafwa Klingan / batterade deras Fiendes
% Klinga och der på söka de att stöt / emedan swagan igenom en battur
% blifwer disordonnerat, och i det samma intet kan [aedera???],
% hwilket wäl kunde lyckas / om icke denna faran woro der hos / och
% den andre i dett samma hade Caverat, och emedan den andre intet har
% rákat hans Klinga  har han dess större Cadur giordt / och har så
% derigenom gifwit sin Fiende ett lyckligare tillfälle att stöta / än
% om han liik en sådan battering eller Klingans nedertryckning erkänna
% ller förändra will / der med när den andra Caverade, har hans Klinga
% på den andra sijdan batterat, hade och detta fara med sig / emedan
% den andre kan förändra sin Cavation och sin udd åter remitera, och
% kunde han altså i den battut intet parera ; Så kan man en häller en
% annars Klinga battera / att man icke derigenom med sin egen kommer
% utur presenza / och så mycket mer när Klingan intet blifwer råkat;
% Offta menar man att dämpa nehr swagan / och går den andra i dett
% samma med sin Klinga så wijdt eller långt för sig / att man i
% stället för swagan finner styrkan / och då batturen ingen effect har
% / kan Fienden sin företagande stöt utan något förhinder fortsättia.
Other are, when they have the Blade, battering their Enemy's Blade and
through this seek to thrust / meanwhile their weak through a beat
becomes disorganised, and in that unable to [aederat???], which could
succeed, if  danger wasn't inherent, and the other in the same had
beaten, and had not reached his Blade he has the larger [Cadur???]
made, and has through this given his Enemy a much more appropriate
moment to thrust / than if he had such a battering or pressing the
Blade downwards acknowledged or changed, thus when the other Beat, he
has his blade on the other side battered, this has some danger, since
the other can change his Beat and bring his point back, and would in
the battering be unable to parry; Thus one cannot another's blade
batter, that one doesn't bring one's own out of
presence\sidenote{Presenza} and so much more when teh blade is not
reached. Often one mean to dampen the weak, and if the other goes wide
or long, that one insted of the weak finds the strong, the battering
has no effect, the Enemy can continue his thrust without any impediment.


% Sådanna här äro de bäste och säkraste / att du din Klinga således
% porterar / i ett tempo parerat och stöter / och på din Fienden ( i
% dett han med sin kropp nalkas) råkar  elliest kan han med en hast
% wijka undan eller bryta Mensuram / och om du skulle förföllia honom
% hade han tillfälle att parera och åter att stöta / och der dett
% skulle hända sig / när du förer din Klinga deportert / att din
% swaga blef batterat / lär han då med en hast blijfwa frii: Din
% styrka måste blifwa orörlig fram för dig / och när du fechtar med
% Avancerter Klinga att du din styrkia utaf din Klinga till defensa
% behålla kan/ och som han dig innan till något hårdt hade
% stringerat och wille der på
The best and safest are thus, that you port your Blade, in a single
tempo parry and thrust, and on your Enemy (in that he with his body
gets closer) reach, otherwise he can quickly void or break Measure,
and if you were to pursue him, he would have an opportunity to parry
and again thrust. And if it should happen, that before you have
deported your BLade, your weak is battered, we will rapidly become
free. Your strong must become unmovable in front of you, and when you
fence with a Forwards Blade, the strong of your Blade can be kept for
defense, and as he has bound you on the inside and then wanted