\newpage

\scan{022}

\newpage

% Translation below

% Innan till är den säkraste med hele kroppen / och Klingan i Qvarta;
% med korset går man temmeligen högt / särdeles om du will passera
% under hans Klinga dinkos / derföre geer da honom tillfälle at stöta
% högt / får du altså bättre tillfälle att passera inder hans
% stöt. Utantill kan du så småningom som du ypnat honom skränka;

inside is the safest with the whole body / and the Blade in Fourth;
with the cross fairly high / especially if you want to pass under his
Blade [ dinkos\sidenote{right? left?} ] / thus you give him an
opportunity to thrust high / you thus get a better chance to pass
under his thrust. You can also threaten him on the outside, where you
have opened him up.

% Men hos alla Chiamater måste du distancen wäl confidera och grant
% taga i acht / att din Fiendes klinga är dig icke för när / att han
% intet råkar dig i det samma du giör din rörelse / / derföre skall
% man ingen Chiamat giöra i den enga Mensura, så kan du och aftaga af
% denne Distancen, och dett är bättre / att du ( i dett samma din
% Fiende griper dig an ) går fort eller retirerat dig / der med du får
% tempo att parera och tillika att stöta;

But with all invitations you must be confident in distance and take great
care that your Enemy's blade is not too close, so, that he can reach you
the moment you move. Thus you should never invite in narrow Measure, so
you can take away this Distance, and it is better (in the moment your
Enemy attacks) go forth or retire, thus gaining tempo to parry as well
as to thrust.

% Derföre är dett intet godt ( när du begynner giöra Chiamater ) att
% du rörer dine fötter / emedan denne Motion är den långsammeste/ och
% satte du dig i fara / om du skulle röre dine fötter / effter såsom
% du hwarken kan komma fram eller tillbaka uti larga mensura, utan
% kropsens rörelse kan man wäl löffta den högre foten; Men den wänstre
% måtte stå stilla / dett samme din Fiende stöter / är du mycket
% ferdigare att stöta contratempo. Merck det hos om din Fiende
% förföllier dig på en Chiamat, eller eliest angriper dig på något
% sätt/ och uti begynnelsen står med föttren så långt ifrå hwar
% annan / så gee gran acht uppå / när han sätter sin wänstra foot
% effter / du i dett samma stöter / emedan han sin högre foot till
% contratempo icke bruka kan / din högre foot kan du först hálla
% sospeso in aria, och sedan låt kroppen med hast föllia / effter som
% kroppen är den hastigaste motion.

Thus it is not good (when you start using invitations) to move your feet,
since this Motion is the slowest, and places you in danger, if you
move your feet, since you will neither be able to move from or back to
Wide Measure, without moving the body you can lift your right
foot. But the left must stand still, in the moment your Enemy thrusts,
you are much more ready to thrust contratempo. Mark well that if your
Enemy pursues you on an Invitation, or otherwise attacks you in any way, and
starts with his feet widely apart, pay close attention to that when he
moves his left foot, you should in that moment thrust, since he will
be unable to use his right foot for contratempo / your right foot you
must hold hanging in the air\sidenote{sospeso in aria}, and then let
your body quickly follow / since the body is the quickest motion.

% Och desse Chiamater skall man göra / då man hafwer att beställa med
% en Fiende som geer Resolut stöter / måste man gifwa honom meer
% lägehet ; En en sådan begiärligheet / betager honom/ att han intet
% blifwer warje den fara som under denne Chiamat war fördölgd; Men
% skulle han dett samma blifwa warse/ kunde han lätteligen förföra sin
% Fiende igen; Och går här list emot list.

And these invites you should do, when you are dealing with an Enemy
that gives Firm thrusts / you must give him more room. Such a desire
will take him, that he will not become aware of the danger that was
hidden under this invitation. But should he become aware of this, he could
easily tempt his Enemy again. And here wit go against wit.

% Om du nu merker / att din Fiende hafwer något i sinnet att
% practicera på dig/ giör du wäl och rätt / att du hiälper och geer
% honom tillfälle der till/ än att du håller honom der ifrå/ och är
% bättre/ att du weet hwad din Fiende will giöra/ låtandes honom dett
% fullborda; Än att han giör något / som dig obekant är/ och du intet
% war betänkt på; Som offta händer sig / att en blifwer stötter och
% weet intet huru det geck till. Alltså är nödwändigt / att man skall
% wetta / hwad en Fiende giöra will/ huruledes du bör bemöta honom / i
% Tempo offendera, ocg dig Salvera.

If you now notice that your Enemy has something in mind, to try on you, you
would do more right, that you help him and give him an opportunity,
rather than try to keep him from it, and it is better that you know
what your Enemy is about to do, and let him complete it. Than him
doing something that is unknown to you, and had not considered. As
often happens, that one receives a thrust and knows not how it
happened. It is thus necessary, that one should know what an enemy is
about to do, how you should meet this, in Tempo offend, and save yourself.

\chap{What Feints are.}

% En Fint är den/ när lossar stöta pa en ort/ och går på en annan 

A feint is that / when you pretend yo thrust in one place / and go
towards another
