\newpage

\scan{061}

\newpage

% Translation below
% hans förehafwande skulle slå honom feelt / lär han intet sedan kunna
% giöra någon pardite: Derföre skall du intet girera, du åst då
% tillförende kommen så nähr i mens' / at du med det förset steget
% kommer bak om hans udd / elliest kan han lätteigen råka dig i ryggen
% eller i halsen / och så snart du hafwer girerat med den wenstre
% foten / rör du dig med en hast den högre / och går sålededs fort in
% till hans kropp / der med du aldeles kommer utur presenza has udd /
% så att han inten kan draga sin Klinga mer tillbaka och giöra dig
% någon skada. Men i fall han hade passerat, giör du intet mer än
% blott wänder kroppen och kommer utur presenza.

his actions should strike him wrong / he would not then be able to do
any [pardite] Thus you should not turn, you are then briught so close
in measure, that you with the first step come behind his point,
otherwise he can easily reach you un the back or the throat, and as
soon as you have turned with the left foot, you move with haste the
right, and go thus quickly in to his body, so you get out of presence
of his point, so that he cannot draw his blade further back and injure
you. But in the case he had passed, you do no more than simply turn
your body and come out of presence.

% Qvarta med Cavation till ett utwärtes tempo, kan du ehrnå af din
% Fiende/ när han tager effter din fint / som giörs med lijtet Järn /
% och en jämbd arm / och han parerar til sin högra / eller träder
% uthan till in på dig / at söka din swaga / så giörs en behänding
% Cavation till din diffesa rätt betäkt hoos hans swaga förblifwandes
% / och giör med udden allena effecten, avancera så / när cavation är
% ändat / det måtte stöten hafwa råkat.

Fourth with a disengage to an outside tempo, can be achieved from your ENemy, when he takes your feint, which is done with little Iron / and a straight arm, and he parries to his right, or steps on the outside in on you, to seek your weak, a handy disengage is done, your defence cover by his weak remaining, and do withthe point only the effect, advandce thus, when the disengage is finished, the thrust have\sidenote{The original has an amplifying word, ``måtte'', but...} reached.

% Men hade du i Cavation något mer jern at bringa igenom / som till
% exempel, när du utan till öfwer Klingan med den avantage aff
% Contrapostur hafwer giord någon Fint, eller Fienden i din halfwa
% swaga / utan till din Klinga instötte / måste du i qvart' hans
% Klinga Falcera, och såledz gå till de gifne blotten.

But if you in the disengage more iron to bring through, for example,
when you on the outside over the BLade wit hadvantage of
contraposition have done a Feint, or the enemy in your half weak, on
ther outside of your Blade thursted, you must in fourth his Blade
Falcade, and thus go to the given openings.

% Falcering är en igenomwriding med Klingan / der uti handen i samma
% situ förblifwer / och du till den ändan ditt kors tillförende i en
% wederbörlig högd emot hans Klinga förer / och kan altså med den
% samma ( om han redan har funnit din Klinga utanb till ) lädera honom
% i hans högre sijda; Och der han inten skulle öppna sig / Caverar du
% med din udd öfwer hans Klinga / förblifwandes och i stötet med
% korset hos hans Klinga; måtte du i fall effter lägenhet ophöja
% korset eller sänka det / så giör kroppen rätt lijten; Men wore hans
% klinga tämmelig hög / blifwer du under den samma med korset wäl
% betächt och stöter altså med wriden Klinga hos hans swaga / åth de
% nehrmeste blott.

Falcading is a through-turning of the Blade, wherein the hand remains
in the same place, and you to that end your cross in a suitable height
against his blade bring, and can thus with the same (if he has already
foudn your blade on the outside) injure him in his right side. And
should he not open himself, you Disengage with your point over his
BLade, and remain in the thrust with your cross by his Blade, you must
depending on circumstances elevate the cross or lower it, the body
does relatively litte. But were his blade high, go under it wit hthe
cross well covered and thrust with e turned blade by his weak, to the
nearest opening.

% Ibland blifwer och qvarta (när Fienden utan till igenom det steget
% p. Qvadro lätt förföra sig ) derstädes instött under hans högre Arm.

Sometimes the fourth (when the Enemy on the outside, through the step
p. Quadro was seduced) thrust under his right Arm.

% Så kan man och med denne stöten förhindra den myckne caveringen i
% det man utan till strin' / och emot hans första ingenomgående
% Contra-caverar tillijka; Man avancerar på Finte-wijs eller sätt, och
% förr än han andra gången caverar / med försänkt udd i qvart' hans
% Klinga med din styrka innan till bringa' och med korset i ett arbete
% öfwer hans KLinga förblifwa med hans swaga / och stöta till de
% underblotten / och honom i hans genomgång opprimera, Hwilket och sig
% emot den utwertes Finten hos korset låter giöra; Så och när du emot
% den mutering innan till under hans Klinga geck igenom och han i det
% du wilt utan till gå fort / åter tänkte cavera; Eller der han med
% sin Klinga i en heel Circel ofwan effter ifrå sin högre till sin
% wenstre muterade / i det han sin Klinga innan till under sig bringar
% / går du utan till fort / och stöter till hans högre höfft

This thrust can also be used to prevent much disengagement in that you
bind on the outside, and against his first through-going
Counter-disengage. Then advance in a Feinting manner, and before he
disengages the second time, with lowered point in fourth his Blade
with your strong bring to the insideand with the cross in one work
over his blade remain with his weak, and thust to the lower openings,
and him in his through-passing press down. This is also against the
outside Feint with the cross doable. And also when you against the
mutation on the inside under his Blade went through and he in this
wants to go outside quickly, again to disengage. Or when he his BLade
in a full Circle above from his right to his left mutated, in that he
his Blade inside and under bring, you go outside quickly and thrust
his right hip.

% Så låter sig och qvarta anbringa till de underblotten / när hon uthi

Thus is fourth applicable to the lower openings, when she in
