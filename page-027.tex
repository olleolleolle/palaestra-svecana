\newpage

\scan{027}

\newpage

% Translation below
% med en hast / för än han har fått Mensuram strictam stöta / kan du i
% dett samma / när han rättar sin udd till din kropp gå med din
% Klinga effter hans och stöta innan till tillika med hinom
% Contratempo, emedan hans Klinga i dett han wänder henne till din
% kropp och utsträcker blifwer förswagat / han utan dett / när du
% befinner dig med ditt fäste i den högden jämt med din Axel / måtte
% han igenom din styrka stöta din Klinga; Om du på ett så hårdt
% string' caverade, måste han tillförende / för än han kan stöta i
% cavation föra sin Arm i dett förrige ställe / hwar af han kan med
% Comittere dispada, eller i den enga messura, genom Messia cavatuin
% lättelgen blifwa stötter; Hwilket och i förste tempo hans string'
% med en cavation har blifwit giordt.
quickly, before he has strict Measure thrust / you can in the same /
when he straightens his point to your body go with your blade after
his and thrust on his inside in Contratempo, since his Blade, in that
he turns it towards your body and extends, will become weakened. He
will, when your hilt is at the height of your shoulder, need to thrust
through the strong of your Blade. If you on such a hard bind disengage,
he will need to before he can thrust into your disengage return his Arm to
the previous position, whereby he can with Committere dispada, or in
narrow measure, through Messia cavatuin\footnote{Messia cavatuir? Mezza-cavation?}
easily be thrusted. Which in the first tempo his bind with a disengage has
been made.

% Så är dett och nödwändigt/ att du så wäl i stringering som stöten
% blifwer patron öfwer din Klinga och förer henne på Finte sätt/ der
% med om du skulle blifwa på en motion förhindrat / att du låter fara
% den samma och söka en annan igen / så lär du din Fiende ( i dett
% han söker att förhindra dig) råka / utan någon drögzmål tillbaka
% dragandes udden / utan till hans kropp fortfara måste / hwilket
% ordenteligen fodras / medan du i stöter . finter, Cavationer, eller
% elliest uti någon motion oppehålla udden / är du intet komma a
% tempo; Man skall och tillförede utsträcka Armen / och straxt der
% uppå i ett tempo falla med kroppen fram för sig; En på sådant
% sätt råkar stöten  för än Fienden blifwer dett rätt warse; förty om
% man med kroppen skulle först falla för sig / blefwo han dett
% letteligen warse / och har så kunnat giordt ett Contratempo der
% emot. Och skall man i stöten a pie fermo med den högra foten de
% adjuto med den wänstra gee / lijka som stötte den wänsra den högre
% foot / blifwer stöten desze hastigare giordt.

It is also necessary, that you in binding as well as thrusting become the
master of your Blade and move it in a Feinting manner. That way, if
any move is hindered, you can abandon it and search another
immediately, you will touch your opponent (in that he is seeking to
deny you), without any delays retrieve your point, but rather to his
body continually must, which is properly required, while you in
thrusts, feints, disengages, or in any other motion occupy your point,
you are not coming into tempo. You should extend your Arm, and soon
after in a tempo let your body fall forwards. In such a manner the
thrust lands before the Enemy is rightly aware, since if you were to
fall forward first, he would more easily become aware, and would be
able to do Countertempo against it. And you should in the thrust be [a
pie fermo] be with the right foot [de adjuto] give with the left, the
same as thrusting with the left the right foot, the thrust is thus
quicker done.

% Detta här kan af dem samme som föra sin Klinga ifrå den ene ohrten
% till den andre utan någon Slancering tages i acht / och är den ene
% rörelsen emellan den andra lätt at erkiänna / och fördenskull när
% Klingan är ferm och stadig / och med fötterna och kroppen blifwer
% wäl öfwerens . på hwilket sätt / de hafwa med styrka och stadighet
% / och den som förer henne / är henne mycket mechtigare/ Och efter
% giord stöt giör han ingen Cadure, utan behöfwer allena med kroppen
% dilingera med foten / i fall han har intet passerat, drar han foten
% med en hast tillbaka igen och på nytt string' sin Fiendes Klinga;
% hände det sig / i dett samma du går tillbaka / han söker att stöta
% dig / eller elliest pröfwar att förföllia dig / kant du med offesa
% och difessa i en jämbd linea gå på honom igen / och desa alla utaf
% den union, der uti du dig med Klingan och fötterna och kroppen befan
% / hwilke observationer om man tager dem wäl i acht / lär dett wara
% säkert att parera och stöta i ett. Ehuruwal i Enfach rapier
% tillhöres ett stort judicium, när man will desze bägge rörelser i
% ett tempo förrätta / der man elliest i stöten skulle giöra twå
% tempo, hwarigenom din Advers' får lägenhet att mutera sin effect.

This can by those who move their blade from one target to another
without any counter-thrust\sidenote{Original says 'slancering'} is
observed, and is when the one movement between the other is easy to
acknowledge, and otherwise when the Blade is firm and steady, and with
the feet and body well in harmony. In which way. they have with
strength and steadiness, and those who move it, is more powerful with
it. And after completed thrust, he does no Cadure, but needs merely
needs to [dilingera???] the body with his foot, if he has not passed,
he can quickly pull his foot back and again bind his Enemy's Blade. If
it so happens that, in the moment you step back, he seeks to thrust,
or merely tries to pursue you, you can with [offesa???] and
[difessa???] in an even line go at him again, and these all from the
union, within which you with Blade, feet and body fidn yourself, which
observations if you take them well in advice, shuld be safe to parry
and thust in one. However, in rapier there is a great consideration,
when one should these two movements execute in a single tempo, where
you would otherwise do two tempos, though which your Adversary is
given opportunity to mutate his effect.

% Alla dina stöter skola med oförwändt Ansichte gå jämpt fram för
% sig / och de linier och steg effter kropsens sänkning / öfwer och
% under / innan och utan till / till din Fiendes högra Axel eller
% sijda råka skalt / effter den wänstra skalt du intet stöta / emedan
% du gifwer dig innan till så mycket yppen / dett wore en ting att din
% wederpart förde wärjan i sin wänstra hand.

All your thrusts should with an unturned Face proceed, and the lines
and steps after the lowering of the body, over and under, ouside and
in, towards your Enemy's right Shoulder or side should reach, after
the left you should not thrust, because you open yourself on the
inside, it were a thing if your opponent carried their rapier in their
left hand.
