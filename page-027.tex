\newpage

\scan{027}

\newpage

% Translation below
% med en hast / för än han har fått Mensuram strictam stöta / kan du i
% dett samma / när han rättar sin udd till din kropp g/a med din
% Klinga effter hans ocg stöta innan till tillika med hinom
% Contratempo, emedan hans Klinga i dett han wänder henne till din
% kropp och utsträcker blifwer förswagat / han utan dett / när du
% befinner dig med ditt fäste i den högden jämt med din Axel / måtte
% han igenom din stryrka stöta din Klinga; Om du p/a ett så h¿ardt
% string' caverade, måste han tillförende / för än han kan stöta i
% cavation föra sin Arm i dett f"orrige ställe / hwar af han kan med
% Comittere dispada, eller i den enga messura, genom Messia cavatuin
% lättelgen blifwa stötter; Hwilket och i förste tempo hans strong'
% med en cavation har blifwit giordt.

% Såár dett och nödwändigt/ att du så wäl i stringering som stöten
% blifwer patron öfwer din Klinga och förer henne på Finte sätt/ der
% med om du skulle blifwa på en motion f"orhindrat / att du låter fara
% den samma och söka en annan igen / s/a lär du din Fiende ( i dett
% han söker att förhindra dig) råka / utan någon drögzm/al tillbaka
% dragandes udden / utan till hans kropp fortfara måste / hwilket
% ordenteligen fodras / medan du i stöter . finter, Cavationer, eller
% elliest uti någon motion oppehålla udden / är du intet komma a
% tempo; Man skall och tillf"orede utsträcka Armen / och straxt der
% uppå i ett tempo falla med kroppen farm för sig; En på sådant
% sätt råkar stöten  för än Fienden blifwer dett rätt warse; f"orty om
% man med kroppen skulle f"orst falla f"or sig / blefwo han dett
% letteligen warse / och har s/a kunnat giordt ett Contratempo der
% emot. Och skall man i stöten a pie fermo med den högra foten nde
% adjuto med den wänstra gee / lijka som stötte den wänsra den högre
% foot / blifwer stöten desze hastigare giordt.

% Detta här kan af dem samme som föra sin Klinga ifrå den ene ohrten
% till den andre utan n/agon Slancering tages i acht / och är den ene
% rörelsen emellan den andra lätt at erkiänna / och fördenskull när
% Klingan är ferm och stadig / och med fötterna och kroppen blifwer
% wäl öfwerens . på hwilket sätt / de hafwa med styrka och stadighet
% / och den som förer henne / är henne mycket mechtigare/ Och efter
% giord stöt giör han ingen Cadure, utan behöfwer allena med kroppen
% dilingera med foten / i fall han har intet passerat, drar han foten
% med en hast tillbaka igen och på nytt string' sin Fiendes Klinga;
% hände det sig / i dett samma du går tillbaka / han söker att st"ota
% dig / eller elliest pröfwar att förföllia dig / kant du med offesa
% och difessa i en jämbd linea gå på honom igen / och desa alla utaf
% den union, der uti du dig med Klingan och fötterna och kroppen befan
% / hwilke observationer om man tager dem wál i acht / lär dett wara
% säkert att parera och stöta i ett. Ehuruwal i Enfach rapier
% tillhöres ett stort judicium, när man will desze bägge rörelser i
% ett tempo förrätta / der man elliest i stöten skulle giöra twå
% tempo, hwarigenom din Advers' får lägenhet att mutera sin effect.

% Alla dina stöter skola med oförwändt Ansichte gå jämpt fram för
% sig / och de linier och steg effter kropsens sänkning / öfwer och
% under / innan och utan till / till din Fiendes högra Axel eller
% sijda råka skalt / effter den wänstra skalt du intet stöta / emedan
% du gifwer dig innan till så mycket yppen / dett wore en ting att din
% wederpart förde wärjan i sin wänstra hand.