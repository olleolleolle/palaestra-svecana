\newpage

\scan{033}

\newpage

% Translation below
% (in)tet mer intendera stöta utan söka defendera sig / i betrachtande
% att han intet kan laedera den andra / desz mindre han gifwer icke sin
% kropp först i faran.

% Är fórdenskull mycket gagneligare och nödigare f"or den lilla / att
% han bemöder sig huru han kan få stringera sin Fiendes Klinga / der
% med när han will gå i mens' intet kan stöta / och när han bracht
% dett så långt / och han seer tempo till stöten / är dett honom
% bättre att han i dett samma passerar, emedan han utan dett /
% Fiendens udd så långt penetrerat har / att han swårligen a tempo
% olederat kan komma tillbakas igen; Dett woro då / att den langes
% Klinga war så långt utur presens, eler hade retirerat henne / att
% han klarligen kan merkia dett han har ett tempo oatt Salvera sig /
% elliest kunde han intet i en motu komma så långt tillbaka / att
% icke den stora / för hans längd skull / hade råkat honom med sin
% Klinga; Men åter på den lillas sijda är och sandt  om skönt den
% store har Liniens avantage, hwar på är mycket angeläget / ändock /
% emedan hans rörelser äro mycket långsammare och giöras med
% större blott / kan han sig intet så wäl utur presens med sin udd
% bringa  / och har fördenskull den lilla mer plasz och mahl att råka
% / när han weet säkert förfoga sig i mens, så wäl att Klingan honom
% bättre Coopererar och intet så stora motioner att defendera sig.

% Altså och att han den störste faran (nembligen udden är passerat och
% emedan lijka fall hans blott "ar mindre och ringare ) står han
% fördenskull i mindre fara / och är fördenskull ala hans werkningar
% säkrare än den störres

% Derföre nár en Starck har att giöra med en Swag / går dett en mindre
% med stor fördehl till; En en sådan skall alt sitt Fechtande läggia
% på string' på Klingan / emedan han lätteligen kan disordinera , och
% på dens rörelse stöta; En när en Swagare skulle resistera emot en
% Starcker / lärde han ( i dett han den andres Klinga söker att röra
% ifrån sig) komma med sin Klinga så långt uhr presens att han
% derigenom lätteligen kunde blifwa stötter / och skedde dett a pie
% fermo, kunde han effter stötens giörande åter gå till hans Klinga
% och hämpta det förriga igen.

% Der emot skall en Swager altijd undfly den andres Klinga / en heller
% låta honom finna sin Klinga / såsom och icke parera när den Starka
% stöter; En dett händer sig offta / att den enes swaga förm/ar med än
% den andres styrkia / och skulle altså den Swaga / i dett han sökte
% parera ; finna sig wara bedragen.

% Om man fördenskull kan hafwa sin Klinga frij / bör man intet parera
% wijd dett / då klingan blifwer mycket beswärat / att han beswárligen
