\newpage

\scan{033}

\newpage

% Translation below
% (in)tet mer intendera stöta utan söka defendera sig / i betrachtande
% att han intet kan laedera den andra / desz mindre han gifwer icke sin
% kropp först i faran.

no more intend to thrust but seek to defend himself / taking into
account that he cannot injure\sidenote{The original, uses l{\ae}dera
rather than lädera, but if we take {\ae} to mean ä rather than ae, we do
get an old Swedish word for injury} the other / with less than giving
his body first into danger.

% Är fördenskull mycket gagneligare och nödigare för den lilla / att
% han bemöder sig huru han kan få stringera sin Fiendes Klinga / der
% med när han will gå i mens' intet kan stöta / och när han bracht
% dett så långt / och han seer tempo till stöten / är dett honom
% bättre att han i dett samma passerar, emedan han utan dett /
It is for this reason more beneficial and necessary for the small /
that he cares about how to bind his Enemy's Blade, through this when he
wants to go into measure unable to thrust / and when he has brought it
tha far / and he sees tempo for a thrust / it is better for him that
he in that moments passes, since he without that / 
% Fiendens udd så långt penetrerat har / att han swårligen a tempo
% olederat kan komma tillbakas igen; Dett woro då / att den langes
% Klinga war så långt utur presens, eler hade retirerat henne / att
% han klarligen kan merkia dett han har ett tempo att Salvera sig /
the Enemy's point so deeply has penetrated / that he hardly can in
tempo uninjured can come back out. It were then / that the tall one's
Blade was so far out of presence, or he's ulled her back / that he
clearly can note that he has a tempo to Save himself /
% elliest kunde han intet i en motu komma så långt tillbaka / att
% icke den stora / för hans längd skull / hade råkat honom med sin
% Klinga; Men åter på den lillas sijda är och sandt  om skönt den
% store har Liniens avantage, hwar på är mycket angeläget / ändock /
otherwise he culd not in opposition coe so far back / that the large
not / for the sake of his height / had reached him with his Blade; But
again on the small one's side it is true that if the large has the
advantage of the Line, where upon is very urguent / nonetheless /
% emedan hans rörelser äro mycket långsammare och giöras med
% större blott / kan han sig intet så wäl utur presens med sin udd
% bringa  / och har fördenskull den lilla mer platz och måhl att råka
% / när han weet säkert förfoga sig i mens, så wäl att Klingan honom
% bättre Coopererar och intet så stora motioner att defendera sig.
sice his movements are much slower and are done with a wider opening /
he cannot bring himself out of presence with his point bring / and for
this sake the small more space and target to reach / when he knows
safely to dispose himself in measure, and the Blade Cooperates better
with him and not so large motions to defend himself.

% Altså och att han den störste faran (nembligen udden är passerat och
% emedan lijka fall hans blott är mindre och ringare ) står han
% fördenskull i mindre fara / och är fördenskull ala hans werkningar
% säkrare än den störres
Thus that he the largest danger (specifically the point is passed and
even though his opening is smaller and less significant) he stands
nonetheless in less danger / and thus all his workings are safer than
the large's.

% Derföre når en Starck har att giöra med en Swag / går dett en mindre
% med stor fördehl till; En en sådan skall alt sitt Fechtande läggia
% på string' på Klingan / emedan han lätteligen kan disordinera , och
% på dens rörelse stöta; En när en Swagare skulle resistera emot en
% Starcker / lärde han ( i dett han den andres Klinga söker att röra
% ifrån sig) komma med sin Klinga så långt uhr presens att han
% derigenom lätteligen kunde blifwa stötter / och skedde dett a pie
% fermo, kunde han effter stötens giörande åter gå till hans Klinga
% och hämpta det förriga igen.
When a Strong has to do with a Weak / it is with less lareg advantage;
Such a one shall in all his Fencing lay binds on the Blade / since he
can easier displace, and on its movement thrust; But when a Weaker
should resist against a Stronger / he should (in that the other's Blade
seeks to move away from him) come with his Blade so far ou of presence
that he though this would be easily thrusted / and if this happened a
pie fermo, he could after the thrust return return to his BLade and
fetch the former again.

% Der emot skall en Swager altijd undfly den andres Klinga / en heller
% låta honom finna sin Klinga / såsom och icke parera när den Starka
% stöter; En dett händer sig offta / att den enes swaga förmår med än
% den andres styrkia / och skulle altså den Swaga / i dett han sökte
% parera ; finna sig wara bedragen.
Against this shall a Weaker always avoid the other's Blade / and never
let him find his Blade / like not parrying when the Strong thrusts; If
this happens often / that one's weak is capable of mor ethan the
other's strong / and would the Weak / in that he seeked to parry; find
himself betrayed.


% Om man fördenskull kan hafwa sin Klinga frij / bör man intet parera
% wijd dett / då klingan blifwer mycket beswärat / att han beswárligen
If one for this reason can have one's Blade free / you shold not parry
wide / since the Blade is very troubled / that he with difficulty