\newpage

\scan{029}

\newpage

% Translation below
% hwar till fodras att man altijd håller fötterna när tillhopa / och
% tillijka med kropsens öfwerböjning den högre handen / och den
% fremste leden strecker man wäl uht / och med Klingan och Armen i
% en jämbd linie, räcka ock streckia måste / och när man då med en
% hast will retirera sig / eller taga foten effter tillfälle tillbaka
% igen / måste man i dett samma intet lyffta kroppen opp utan kröka
% wänstre Låhret/ och på den samma som tillförede wichten utaf kroppen
% Fundera, der med att den högra till oplyfftning och fortsättning är
% desto prompter, hwilket altsammans måste skee i ett tempo.

to which is needed that you always keep your feet near each other, 
and furthermore with the bending of the body, the right hand and its
foremost joint are well extended, and with the Blade and the Arm in an
even line you reach and stretch, and when you want to retreat in a
haste, or to take the foot back after an opportunity, you must not in
that lift your body but just bend your left Thigh, and onto that
transfer and ground the weight of your body, on that the right is
that much quicker to lift and continue, which must all happen in a
single tempo.

% Denna acht att stöta / när man har bracht dett i en god öfning /
% giör han kroppen skickelig / eller benen dispost, lärer wäl att
% judicera distantierna och bringar tillwäga att man giör en stöt
% mycket längre  än som man elliest af naturen giöra kan / derföre de
% och emot retirerad gvardien till sig dragande parader dugeliga
% äro. Item emot stackote Personer säkert kan bruka.

This way of thrusting, when you have brought it to good practice,
makes the body skilled, or the legst [dispost]\sidenote{Not entirely
sure what 'dispost' would mean here?}, teaches to well judge distance
and makes possible to thrust much further than what otherwise be
natural, therefore they and against retreating guard towards oneself
parries are useful. [Item???] against [stackote] Persons safely can be
used\sidenote{This may be ``can be safely used against stronger persons''}

% De äro på detta sätt intet gode/ att man sätter wänstre foten fram i
% posturer; En om man skulle stöta med wänstre foten utan pasade,
% skulle man intet stöta långt / och gingo man in med den efftersta
% foten / och i dett samma stiga tillbakas igen / skulle det för desz
% stora tempo intet lyckas/ utan man kommer så långt in att man i tempo
% intet kan komma tillbaka igen.

They are in this way not good, that one places the left foot forwards
in postures. If one shoudl thrust with the left foot withour passing,
one would not thrust deep, and if one went in with the rearmost foot,
and in the same steps back again, it would for its large tempo not
succeed, because you get so far in, that you cannot return in tempo.

% Är fördenskull de här / och elliest få andra orsker skull som man
% kunde anföra / intet tienligt att man sätter wänstre foten fram /
% dett wari då man wäntade på sin Fiendes stöt / att man i dett samma
% sätter wänstra foten tillbaka och tillijka parera och stöter /
% hwilket wäl låter giöra sig / i dett att högre Axelen tillstöten
% kommer åter fram ; Men om han intet har stött / måtte du intet
% grijpa honom an / utan sätter din högre foot fram / så kan du och
% emedan foten och kroppen giorde mindre Motion, mycket för och
% behändigare stöta / och dig med en hast Salvera; Elliest är dett
% intet illa / att man effter giordan stööt / den högre foten bak om
% den wänstra / och fölian den wänstra bak om den högre sättes / der
% men man åter kommer med den högre foten fram / emedan du kommer så
% långt din Fiende på sådant sätt tillbaka / och i dett samma att han med
% sin stöt intet kan råka dig.

It is for this sake, and otherwise only few sakes that one could
mention, not serviceable to place the left foot forward, though it
could be that when one is waiting for one's Enemy's thrust, that one
in that moment pull the left foot back, and in that parries and
thrusts, which is well done, in that the right Shoulder into the thrust
comes forwards. But if he has not thrust, you must not attack him, but
put your right foot forward, so you can, since teh foot and body has
made a smaller Movement, much handier thrust, and Save yourself
quicker. Otherwise it is not bad, tah after a completed thrust, the
right foot behind the left, and following that the left behind the
right, through this you are again right foot forwards, meanwhile you
get so far back from your Enemy in this way, that he in that moment is
unable to reach you.

\chap{What passing is.}

% The original manuscript elides hapter 13, so we do too.
  \stepcounter{PalaestraChapterCounter}

% Passera heter / när man med både fötterna till sin Fiendes kropp
% ingår / hwilket man som ett tienligit och fördelachtigt ting i
% fechtning nest stöten a pie fermo och måtte wetta att bruka; En der
% igenom blifwer Fienden mer Turberar och uti större räddhåga satter;
% Stötena går mycket starkare / och bewijsar sig mer mod och
% tapperheet der hos / Klingan Kroppen och 

Passing it is called, when one with both feet towards one's Enemy's
body setp, which one as a serviceable and advantageous thing fencing
next after a thrust a pie fermo and must now how to use. Through this
the Enemy becomes more disturbed and in greater fear is
placece. Thrusts are much stronger, and prove more
bravery\sidenote{'mod och tapperhet' could be translated as ``bravery
and bravery'', both words being in the 'brave' spectrum, with subtle
differences, for brevity only ``bravery'' is used} in Blade, Body and
