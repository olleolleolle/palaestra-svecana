\newpage

\scan{029}

\newpage

% Translation below
% hwar till fodras att man altijd håller fötterna när tillhopa / och
% tillijka med kropsens öfwerböjning den högre handen / och den
% fremste leden strecker man wäl uht / och med Klingan och Armen i
% en jämbd linie, räcka ock streckia måste / och när man då med en
% hast will retirera sig / eller taga foten effter tillfälle tillbaka
% igen / måste man i dett samma intet lyffta kroppen opp utan kröka
% wänstre Låhret/ och på den samma som tillförede wichten utaf kroppen
% Fundera, der med att den högra till oplyfftning och fortsättning är
% desto prompter, hwilket altsammans måste skee i ett tempo.

% Denna acht att stöta / när man har bracht dett i en god öfning /
% giör han kroppen skickelig / eller benen dispost, lärer wäl att
% judicera distantierna och bringar tillwäga att man giör en stöt
% mycket längre  än som man elliest af naturen giöra kan / derföre de
% och emot retirerad gvardien till sig dragande prader dugeliga
% äro. Item emot stackote Personer säkert kan bruka.

% De äro på detta sätt intet gode/ att man inan sätter wänstre foten
% fram i posturer; En om man skulle stöta med w"anstre foten utan
% pasade, skulle man intet st"ota långt / och gingo man in med den
% efftersta foten / och i dett samma stiga tillbakas igen / skulle det
% för desz stora tempo untt lyckas/ utan man kommer å långt in att man
% i tempo intet kan komma tillbaka igen.

% Är fördenskull de här / och elliest få andra orsker skull som man
% kunde anföra / intet tienligt hatt man sätter wänstre foten fram /
% dett wari då man wäntade på sin Fiendes stöt / att man i dett samma
% sätter wänstra foten tillbaka och tillijka parera och stöter /
% hwilket wäl låter giöra sig / i dett att högre Axelen tillstöten
% kommer åter fram ; Men om han intet har stött / måtte du intet
% grijpa honom an / utan sätter din högre foot fram / så kan du och
% emedan foten och kroppen giorde mindre Motion, mycket för och
% behändigare stöta / och dig med en hast Salvera; Elliest är dett
% intet illa / att man effter giordan stööt / den högre foten bak om
% den wänstra / och fölian den wänstra bak om den högre sättes / der
% men man åter kommer med den högre foten fram / emedan du kommer så
% lgt din Fiende på sådant sätt tillbaka / och i dett samma att han med
% sin stöt intet kan råka dig.

\chap{What passing is.}

% Passera heter / när man med både fötterna till sin Fiendes kropp
% ingår / hwilket man som ett tienligit och fördelachtigt ting i
% fechtning nest stöten a pie fermo och matte wetta att bruka; En der
% igenom blifwer Fienden mer Turberar och uti större räddhåga satter;
% Stötena går mycket starkare / och bewijsar sig mer mod och
% tapperheet der hos / Klingan Kroppen och 
