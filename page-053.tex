\newpage

\scan{053}

\newpage

% Translation below
% at du går med din Klinga till hans blott / träder du med din styrka
% till hans swaga / och Continuerar på sådant sätt der hos / på dett
% att han intet med en angulo kan råka; Derföre du med din klinga
% somblige gånger i stóten går jämpt fór sig / ibland öfwer och under:
% udden bör wändas / hwar effter Fiendes blott eller swaga går
% högt eller lågt.
taht you go with your Blade to his opening, step with your strong
against his weal / and Continue in this way, so that with an angling he
cannot reach. Thus you should go with your blade sometimes in the
thrust even, sometimes above and below. The point should be turned, as
the Enemy's opening or weak goes high or low.
%                         Och merck att emot de nedrige guard' eller rörelser
% / effter giord ligation med hela secunda, uti en angulo med halfwa
% secunda, försänka korset / och udden wäl opp till hans Klinga / emot
% högre gvard' och rörelsen till större defension, hela secunda
% emädan derigenom Fiemdens klinga blifwer bättre bort parerat, warder
% altså stöten utan till hans Klinga giord; Dett wari då, att du
% tillförende med din Klinga har wistas i qvarta, du brukar då för
% mindre rörelse skull tertia.
And note that against the lower guard or movements, after a bind with
the whole second, in an angle with half second, lower the cross, and
the point well up his lade, against a higher guard and the movement to
a larger defense, the whole second as through this the Enemy's blade
is better parried away, the thrust is thus done outside his blade. It
being such, that when you have with your Blade been in fourth, you
should use, for smaller movement, third. 
%                                            Uti andre fall blifwer secunda innan
% till öfwer eller under klingan med avancerande högre Axeln på bägge
% sätt a pie fermo, eller och med passaden stötter. Hon kan och
% blifwa anbrachter utan till öfwer Klingan / när han på den
% utwärte Finten intet aldeles parerar öfwer sig / att du i dett samma
% förwänder utur tertia i Secunda, på hwilket sätt / du lär
% åter komma med dit kors till hans swaga / och med försänkter udd
% stöta utan till öfwer hans Arm in.
In other cases is second on the insode, over or under the blade with a
forward right Shoulder in both ways a pie fermo, or with the passing
thrusted. It can also be brought on on the outside over the Blade,
when he on the outside Feint doesn't parry completely above himself,
that you in that moment turn out of third into Second, in which
manner, you should again come with your cross to his weak, and with
lowered point thrust on the outside over his Arm.
%                                                      Så kan du och din Fiende (när du
% har emot en under gvardia stringerat / och han will utan till med en
% lång stöt öfwer din Klinga) stöta i secunda contratempo. Utan
% till under Klingan blifwer secunda med eller utan ligade
% stötter. Med ligation läderar man Fienden på många sätt / med
% större säkerhet som till Exempel, när du din Fiende innan till med
% contrapostur, eller med Finten per qvartam är mäst kommen i den Enga
% mens' /
You can also your Enemy (when you have bound against a lower guard,
and he tries to do a long thrust above your Blade) thrust second
countertempo. On the outside under the Blade is second with or without
bind thrusted. With a bind you can injure the Enemy in many ways, with
greater security, like for Example, when you the Enemy on the inside
with counterpoise, or with the Feint in fourth is come into Short measure,
%              han då utan fotens rörelse caverat, eller utan till en
% angulerat chiamata giorde / och du öfwer haus motion stötte in
% öfwer hans Klinga eller Arm / och han giorde i dett samma en volt,
% williandes stöta under din Klinga in / eller söka att råka
% dig tillijka / eller hålla udden för dig / så lät med en hast din
% udd i hehl secunda siunka / och stöt i samme ligation med en
% wridning / förr än han har ändrat cavation uti ett tempo utan till
% under hans Klinga / så lär du honom i hans förtagnde cavation
% opprimera.
he then without the foot's movement disengage, or on the outside make
an angled invitation, and you over hs motion thrusted in above his Blade
or Arm, and he did in the same make a circle\sidenote{``volt''},
trying to thrust in under your Blade, or seek to reach you, or hold
the point in front of you, let with haste your point in full second
sink, and thrust in the bind with a twist, before he has changed the
disengage in one tempo outside under his Blade, you should him in his
denied disengage depress.

% Man kan och i det samma giöra en contracavation / och de offta / aff
% och till gående med klingan / blifwer honom förhindrat.  Men wore
% du tillförende innan till något diupt kommen / så skränk kroppen i
% ligade under stöter / så mycket / att du med ditt kors kan blifwa
% med hans swaga ; Udden rättar du till hans blott / din wenstra Axel
% såsom och wenstre handen förer du wäl fram / så komer den högre
% sijdan bättre sinkos / och du färdigare att taga honom med dit
% wänsta hand wijd hans fäste.
You can also in the same make a counterdisengage, and they often,
leaving or approaching with the blade, is him denied. But were you
come, on the inside somewhat deep, twist the body into the bind unde
the thrust, so much, that you with your cross can be at his weak. The
point you aim for his opening, your left Shoulder ass well as left
hand you move well forward, so the right side is better lowered, that
you more easily can take him with your left hand on his hilt.
%                                                Men skedde denna cavation eller
% chiamata in larga mens' så tag tillförende med ligation bättre
% Mensuram / och stöt med dett samma utan till under hans Klinga /
% eller gack öfwer i en jämbda linie / och föhr klingan på fintwiisz
% / eller sänck till Mensuram och stött der på som förr sagt är /
% med ligaden fort. Och så du wäl här emot desze Chiamater med en fint
% per Tertiam i en jämbd linie med lijtet jern brede wijd hans korsz
% giöra / så kan du igenom en lijten [ motion ] en förr komma til at
% lädera / och har du intet behof att fruchta dig / dett han (emdan
% då finterat med din swaga i hans swaga) utan till skulle stöta; En
% (för än han med sin udd kommer i Presensa och har sin stöt
% förrättadt) hafwer du redan / dig Cavation och stötet tillijka
% libererat / eller och taga Contra tempo der emot.
But if this disengage or invite  happens in wide measure then take with
a bind a better Measure, and thrust immediately on the outside under
his Blade, or go over in an even line, and move the blade in
feint-manner, or lower to Measure and then thrust as previously
mentioned, with the bind quickly. And you can well here against these
invitatons with a feint in Third in an even line with small iron beside
his cross do, you can with a small motion\sidenote{This is conjecture,
the text is quite faded, but it looks plausible}, quickly come to
injure, and you have no need to fear, that he (since you feinted with
your weak in his weak) shoudl thrust on the outside. Now (before he
with his point comes into Presence and have his thrust executed) you
have already, your Disengage and thrust already delivered, or taken
Countertempo against it.
