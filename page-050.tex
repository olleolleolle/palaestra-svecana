\newpage

\scan{050}

\newpage

% Translation below
% till läderat; Orsaken at han intet kan hinna till din underdeel /
% blifwer han då af din Klinga råkader / hwilken är närmare / och
% allredan utsträchter;
injured. The reason he cannot reach your lower body, he will be by
your Blade reached, which is closer, and already extended.
%                                Derföre om du wilt säkert råka honom i denna
% gvard' motte du söka att fånga hans Klinga sin koosz / taga tempo
% i acht / och sedan på den ene eller andra sijdan ./ med din kropp
% falla undan hans udd / och til hans kropp bort passera / emedan du
% intet kan råka honom / du har då tilförende passerat hans udd;
Therefore if you want to securely reach him in this guard, you must
seek to catch his blade, take tempo in consideration, and then on one
side or the other, with your body void his point, and to his body
pass,while you cannot reach him, you have already passed his point.
% Effter du intet kan Salvera dig / i det du går tilbaka / är
% fördenskull bättre att du fulbordar din påbegynta gång som skier
% med passade; Elliest måtte man i denna guard' observera / att du
% håller henne långt och wäl utsträckt / der hoos att giör små steeg /
% hwar med att de underblottem blifwa wäl sin kosz tagne;
After you cannot save yourself, in that you retreat, it is thus better
that you complete your started step that happens with
passing. Otherwise you must in this guard observe, that you hold it
long and well outstretched, take small steps, in which the lower
openings are well absent\sidenote{Technically, the original seems to
say ``the lower openings are well gone''}.
%                                                                                            Och i fall
% att Fienden skulle så hårdt nalkas att du bättre kan retirera dig /
% således / att i denna guard' fordar nöden / dett du håller honom
% longt ifrån dig / elliest kunde han lätteligen få tilfälle at
% passera in på dig. För hwilken orsak / att somblige med Klingan
% intet så mycket Avancera / giöra de derföre / att att de befruchta
% sig / dett dera Klinga blifwer dem förr funnen och String' / hålla
% icke desz mindre Armen raak / men udden försänckt; Hwilket när det
% är alstå exercerat / och occasio för handen / går denna rörelsen
% desz iuster och mycket hastigare; Hwar emot de som intet hafwa want
% sig på desze läger / lära och intet wara sin diffesa förwisz och
% säker / än som han hade exercerat.
And in case the Enemy should close so hard taht you better retreat,
thus, in this guard necessity demands, that you hold him far from you,
otherwise he would get an opportunity to pass in on you. For this
reason, that some with the Blade do not much advance, therefor they
make themselves, fileld with fear, that their Blade becomes quickly
found and bound, holding the Arm nonetheless straight, but the point
lowered. Which when it is then exercised, and [occasio] for the hand,
goes this movement [iuster] and much quicker. Against those who have
not become accustomed to this position, you should not your defence be
too sure and sure, than if he had exercised.

% Hwad anbelangar de samme som hålla Klingan i tertia farm för Knäät /
% eller i en anguleret secun /de hålla sin Klinga i handen mycket
% fastare och starkare; Men Kroppen blifwer derigenom mycket blotter /
% och gifwer Fienden förmedels detta fast mer tillträde; Ibland funne
% de samma / som liggia uti detta läger i Tertia, intet Cavera, af
% orsak at udden är till högden angulerat, giöra fördenskull alt för
% stort tempo / derföre du och hans Klinga med en battute disordinera
% kan / så at han i din stöt intet kan angulera.

As concerning those who hold the Blade in third forwards of the knee,
or in an angled second, they hold the Blade in the hand much firmer
and stronger. But the Body becomes through this more open, and gives
the Enemy through this much more access. Some times those, who use this
position in Third, not Disengaging, because the point is upwards
angled, doingthus a too large tempo, therefor you can his Blade with a
beat disordinate, so that he in your thrust cannot angle.

% Med en Angulerat secunda: kan man Cavera och lädera; Men sig der med
% defendera, är den der till odugelug; Särdeles emot den / som igenom
% swagan i en jembd linea weet att stöta , och i det samma ofawn
% till blifwer Coopert / gåendes med korset högt / så framt han hade
% welat parera, måtte nödwändigt giöra en stor Motion, och få så
% medelst offta intet tempo;
With an Angled second, you can Disengage and injure. But to defend
oneself, it is useless. Especially against those, who through the weak
in an even line know to thrust, and in the same above become
[Coopert???]\sidenote{Again, this could be a corrupted ``envelope''?},
going high with the cross, so that if he had wanted to parry, must by
necessity make a large Movement, and thus frequently does not get tempo.
%                                           Och der dett skulle än hända / gar dett
% dock med sådan tardite till / att den andre får occasion sin effeck
% genom en annan lineam rectam eller angulum att mutera; Ty en angulus
% går wäl igenom den andre; Men de rätte linier möta hwar anna intet
% så/ utan hafwa de lijka force /så aflöpa de bägge förgäwes; Men
% råkar den ene den andra / kommer dett der utaf att den ene den
% andres swaga bättre ha occuperat / och fördenskul är starkare;
And should that nonetheless happen, it wil be so slow, that the other
gets occasion to mutate the attack through another line or
angle. Since one angle go well through another. But straight lines do
not meet each other so, if they have equal force, they are both in
vain. But if one reaches the other, this  comes from one having
occupied the other's weal better, and is thus stronger.
% Blifwer derföre den swagaste altijd utur linien emporterat / den
% andra går iempt för sig och råkar; Der emot passerar den ene angulum
% genom den andre / och råkar utan resistance af den andre; ja som
% mera är / Cederar den ene den andre / mycket snarare råka de / och på
% sådant sätt passera de emot de jämbde linier och träffa således
% wijd Fiendens rörelse; Så komma och de angulerta stöter långsam
The weaker is thus always pushed out of line, the other goes evenly
and reaches. On the other hand one angle passes through the other, and
reaches without resistance the other. Furthermore, if one deflects
from the other, they reach even quicker, and in this way they pass
against even lines and thus hit in the Enemy's movement.