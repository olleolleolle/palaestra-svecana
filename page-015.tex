\newpage

\scan{015}

\newpage

% Translation below
% Cap 5
% På whad sätt man skall gå sin Fiende an och med försichtighet
% stringera honom hans klinga
%
% Uti angången och string' sin Fiende skall man intet med någon furie
% komma ut mensura utan så småningom begynna först sättia den högre
% foten någon fram och låta så kroppen effterföllia sachteligen fram
% öfwer dett högre låret drag altså den wenstre foten oförmärkt der
% effer;  Sedan sätt den högre foten i sitt ordinarie ställe; Ditt
% fäste förer du jämpt din högr sijda fram wijd samme högd din Axel /
% med hwilken högd så wäl i larga som stricta mensura effter
% möjeligheten är at maintinera; See och till att du får hans swaga /
% dock lijktwäl moste man försichtigen härutinnan ( effter lägenhet af
% sin Fiendes distance och postur) procedera; En understundom är hans
% Guard så retre, att när man utan kropsens förstrekning / allenan med
% swagan wille gå [frå ???] till sin udd / att han då snrt i
% begynnelsen med fara låret räta uti den Enga mensura, som du uti de
% andra retirerade under guard, see kan uti hwilka, och din Klinga
% öfwer hans på den ort hwart uth udden länter sig / är som starkast
% skall hon altijd i den utwärtes Contraposstur öfwer hans swaga / att
% hon lijtet synnes uht till hans wenstra; Men innan till kommer hon
% så mycket blott att stå / att der med din Fiendes swaga blifwer
% betäkt. Uti detta måtte du i string taga wijd acht / att du intet af
% för mycken begiärelse söker wara den starkeste med din Klinga / att
% du icke kommer der igenom din Fiende så långt i mens / ty om han då
% hade tagit tempo i acht hade han lätteligen kunnat råka dig; Du kan
% och tillförende för än du string med udden gå i prospective till
% hans kors / och derpå stringera, sampt med en hast stöta till hans
% blott. Hwad mera beträffar string är allredo uti dett 3 Cap omtorr
% med hwad hårdhet somblige sokia att stringera.

\chap{In what manner one should attack one's enemy and with care bind
his blade}

In the approach and binding of your enemy you should not furiously
come out of measure, but place the right foot slightly forward and
slowly let your body follow over your right thigh, then let your left
foot close. Move your grip even with your right side, at the
height of your shoulder. You should maintain this height in Wide as
well as in Narrow. Also ensure that you get his weak, although you
must proceed with care (taking your Enemy's distance and posture into
account).\marginpar{The next few sentences make no sense in the
original Swedish, so the translation is as close as I can make
it...} Occasionally his Guard will be so retreated that you can without
stretching go  to the weak with the point only, he can
with some danger enter Narrow Measure by straightening his thigh, like
you in second retire under guard, and you blade over his in the
direction whereto the point lends itself, the strongest  she will
always in the outside Contrapostur over his weak, so as to be only
barely visible on his left. But ensure that that you are not inwards
enough that your enemy's weak provides cover. In this you must take
into account that you are not so eager to be the strongest that you end up so deep in your Enemy's measure that he can
easily reach you; You can, before you bind, go to his cross and
thereafter bind, then quickly thrust into the opening. As binding is
further concerned, we already speak in Chap. 3 about how hard
some seek to bind.

% När du nu effter desze regles och avertimenter hafwer string' din
% Fiendes Klinga han i dert samma / när du will string' hans Klinga
% han då utur en öfwer i en under eller utur en under i en öfwer
% gvardia geck / och intet wore ferm med sin Klinga och blifwer i dett
% / aff din Klinga befrijat / moste du på nytt söka att string hans
% Klinga ty ju nermare du befinner dig med din Klinga hins hans / ju
% bättre och säkrre är det; men du måste achta dig / att i dett du
% söker finna hans swaga / han di samma tempo med sin Klinga far fram
% för sig / och du i stället att få hans swaga / råkar uti hans styrka
% ; och om han i det samma du will finna hans Klinga allenast med en
% angulo ifrån dig geck sinkos emellan bägge Klingorne är en ringa
% distance; En så framt medan du hade mens' stötte / skulle du med din
% sturkia av Klingan så lågt komma fram / att du med den samma intet
% kunde mer defendera; Sårdeles när han tillijka med kroppen din stöt
% undflyr och låter gå mistom; Eller kunde och i det du will finna
% hans Klinga / han i dersamma kan pasera, för än du får bringa din
% udd åter i presensa.

When you have after these rules and warnings\marginpar{The original
has ``avertimenter'', as far as I can tell this means ``warning'' or ``information''} bound your Enemy's Blade
he will at the same time, as you try to bind his Blade, go from a high
to a low or from a low into a high guard and if you are not firm with
your Blade and your blade is freed, you must again seek to bind his
Blade, since the closer you are with your blade to his, the better and
safer it is. But you must take care, that when you seek his weak, that
he in the same tempo extends his Blade and you instead of his weak
finds his strong. And if he in that instant you are trying to find his
blade only angles away to your shoulder between both blades is a
small distance; One so forward that you were in distance at the
thrust, you would end up with the strong of the blade so low that you
would not be able to defend with it. Thus when when he retreats from
your thrust with his body, you can leave it. Or you could in that you
want to find his blade, let him pass so that you can bring your point
back in presensa.

% Nu alt sådant att undfly / så merk fölliande General Regel emot
% desze lijke mutationer. 1 Moste du den distancen emellan edre bägges
% kroppar och Klingor wäl betrachta och när du will finna hans Klinga
% / att du intet kommer diupare än i Larga mens och med fötterne ( i
%  det du rörer Klingan till string' ) står ferm; Men wore du intet i
%  hans mutation fullkomligen i Mens' och wille bringa din fot i den
%  samma / moste du förut med Klingan föra en Contrapostur, sedan
%  först sättia foten och så länge med Klingan åter wara
%  stilla. Korteligen att tala  eemdan an i strng' icke som i finterna
%  / med udden is presensa, utan mestendels utur den samma går / motte
%  du intet gifwa Fienden tempo med Klingan och fötterne tillijka /
%  förblifwer du i den linie, der uti du din Fiende åft angången
%  för [honoö ???] och giörs intet behof / när han sin Klinga
%  angulerar, att du trädet utur linien, utan går med din Klinga a
%  traversion till hans swaga / allenast så mycket att hon blifwer
%  derstdes Coopert 3. Så gack ed din Klinga rätt långsamt utan någon
%  violence af Armens rrelse / allenast med kropsens hiälp och den
%  fremsta lede af högra handen / till hans klinga. Så at om han i din
%  rörelse wille något tentera att du åft din klinga mechtig och uti
%  en geswindheet med honom uti ett tempo låta den påbegynte rörelsen
%  fara / och taga Contratempo der emot.

To escape from all of this, mark this following General Rule against
changes like this. 1. You must well mark the distance between your
bodies, as well as that between your blades and that when you want to
find his Blade you should never go deeper than Wide and with your feet
stand firm (when you move your blade to bind); But if you are not in
full measure in his mutation, and want to bring your foot into that,
you must first move your blade into Contrapostur, then move your foot,
and keep the blade still. Briefly, while in a bind, unlike in feints,
your point is not presented, but mostly out of that, you must not give
your Enemy tempo with both blade and feet. If you remain in the line
where your Enemy's blade and feet for his usual move, and no need is
present; when he angles his blade you should not step out of line but
traverse your blade to his weak, only enough that it there becomes
lodged. 3. So your blade should go slowly to his blade, without any
violent arm motion, but only with assistance from your body and the
foremost joint of your right hand. So that if he in your movement
wants to try, you can, with your mighty and swift blade, let him
continue the movement for a tempo, then take counter-tempo against it.
