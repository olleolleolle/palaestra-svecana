\newpage

\scan{017}

\newpage

% Translation below

% jämpt och wäl gå till din Fiendes högra sida / der medd han intet
% utan till under din klinga kan instöta / det ware sig at du dersädes
% giorde en Chiamat, och han i det samma stötte / du med quarta togo
% innan till contratempo. I det öfrige måste det samma taga i acht /
% hwad tillförende hoos contrapostur är omtalt och påmint; Om du nu på
% det sättet di Fiendes Klinga genom sogget[???} string hafwer giordt
% henne underkastat / hwilket är en begynnelse till victorie, och weet
% at behålla din Klinka frij / lär han intet kunna stöta dig / utan
% sig snarare retirera än approchera, så länge till des han seer sitt
% advantage. I fall han med sin Udd i en continuerlig rörelse wore /
% eller för dig i en halff eller heel Circel med stadig arm muterade,
evenly and well go to your Enemy's right side / thereby he will be
unable to thrust beneath on the outside of your blade / it were that
you there made an Invitation and he thrusted at the same time / you took
contratempo from the fourth. You must also pay attention to what we
have said about contrapostur; If you now in this way have bound your
opponent's blade into submission / which is the beginning of victory,
and know to keep your Blade free / he will be unable to thrust into you
/ and will rather retreat than advance, as long as this is
advantageous. In case he moves his point continuously, or in a full or
half circle moves it with a steady arm,
% och du hans swaga ofwan till intet wäl kunde stringera / utan
% der eöfwer med din Klinga kunde blifwa disordonnerar[???] och
% stötter / skall du i ställe / när hans klinga står till hans högra
% ut / med udden innan tll jämpt till hans kors som det centrum, der
% hans kling sing minst rörrer tillgå; men förer han Klingan i mutering
% till sin wänstra / s/a gack utan till i en jämn linea til hans kors
and you are not well able to bind his weak from above / without there
over with your blade get displaced and thrusted / you should instead /
when his blade is outwards to his right / with the point on the inside
smoothly to the cross with the centre being where his blade is moving
the least, but if he moves his Blade ina mutation to his left, so go
outwards in a smooth line to his cross / 
% / rörer han klingan tillijka med Armen / så gack ti hans högre Axel;
% Din udd blifwer så wäl utan som innan til ( när du på det sä8tt
% tilträder hans kors eller axel) förd uthi en halff qvarta,
% undantagandes när han med sin udd ligger lägre än ditt Bälte / d/a
% skeer det i Tert. på hwilket sätt han blifwer twungen att liggia
% stilla med sin klinga / eller kommer då så diupt in / at du honom i
% en s/adan mouvement stöta kan / och han sedan effter sin lngsamme
% förwäntn intet kan parera.
If he also moves the blade with his Arm / go to his right shoulder;
Your point will be moved as well on the outside as on the inside (when
you in this fashion enter his cross) in  half fourth, except when he
has his point lower than your belt / then you should act in Third, so
he will become forced to still his blade / or go so deep / that your
can thrust into his movement / and he after slow anticipation cannot parry.
% Men emoot Secunda och Prima, går du under
% hans klinga i en jämn linie per Quartam till hans kors; Skulle han
% nu med sin klinga ligga stilla / kan du honom den samme ofwan til
% string. eller der han intet wille liggia stilla / kan du utan til
% ifrå hans sida giöra en ligade. Ehuruwäl den rörelsen men en hel
% Circel skeer / kant du på fölliande sätt ofwan til wäl
% förstängia.
But against First and Second, you should go below his blade in a line
to the fourth into his cross; If he then lets his blade lie still, you
can bind from above, or if he doesn't / you can bind from
outside. However the movement with a full circle is done, you can
close above in the following way. 
%     Giff acht uppå til hwilken sijda circulen ofawn til
% omkringgår, Går han ofwan effter ifrån sin h¦ogra till sin Wenstra /
% så gack med din udd jämpt till hans kors såsom de centro, då låter
% han ingen gång röra dun kilnga. Ophäff med en hast din udd / i det
% hans klinga kommer ofwan effter omkring / och wändt honom innan
% till, så har du funnit hans klinga innan til, stringera altså; men
% gieck hans klinga ofwan til ifrån hans Wenstra till hans Högre / så
% uphäf din udd / at din klinga blifwer något uthwärtes wänder / då
% hafwer du henne der med stringerat. Giff altså wijdare acht uppå din
% Adversarii motioner, hwar effter du ditt arbete kan rätta och
% bestyra.
Pay attention to to which way the top of the circle moves. If it is
from his right to his left / go with your point smoothly to his cross,
which is the centre, then he will not touch your blade. Upset your
point hastily / in that his blade will come from above / and turn him
to the inside, and you have found his blade on the inside, thus
bindining; But if his blade moves from his left to his right / upset
your point / so that your blade is slightly turned to the outside,
then you have bound his blade.Furthermore, pay attention to your
Adversary's movement, and adjust and correct your work after that.

\chap{On tempo and contratempo}

% Tempo kallas här den rörelsen / som din Adversarius giör uti
% mensura och dig igenom den samma oundwijkeligen / emedan han intet
% kan giöra twå rörelser på en gång / så att du får en ypning till att
% stöta / heller och elliest en occasion derigenom winner / som är dig
% till en fördel; Lijkwäl skall den motion att stöta / intet wara
% större och längre än tempo är / som dig blifwer gifwit utaf din
% Adversarius. En elliest på sådant sätt att han fådt tiid att parera
% / förr än du har hindt honom / och kunnat satt [ fatt ??? ] dig
% imedlertijd uti fahra. Der emot om du blifwer warje din Fiendes
% rörelse / l"arer din stöt utan twifwel lyckas.

Tempo is here the name for the movement tht your Adversary does when
in measure and unavoidably through you / since he is unableto do two
movements at the same time / so that you get an opening to thrust /
more preferrably an occasion that wins you an advantage; Nonetheless
the movement in the thrust shall not me longer or larger than the
tempo given to you by your Adversary. Otherwise, he will have time to
parry before you have hit him and place you in danger. If you are
aware of your Enemy's movement, your thrust will surely succeed.