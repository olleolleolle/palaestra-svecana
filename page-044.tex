\newpage

\scan{044}

\newpage

% Translation below
% Men när Fienden i runden går omkring och söker mycket mehr med
% stegen / än med Klingan att förföra dig / så blif emot honom lijka
% som i Centro, då kan du för genom en lijten motion dig prospectiua
% förr än din Wederpart med tu steeg i Circumferencen wända / behålt
% altijd din udd frij / och rätta honom till din Fiendes kropp; Dine
% fötter hålt wäl tillhopa / så blifwer rörelsen uti wendningen des
% mindre / och tag Mensuram och tempo wijd dett samma i acht.

But when the Enemy walks in the round and searches to seduce you more
with steps, than with the Blade, be against him in the Centre, then
you can with a small motion prospect before your Opponent can with two
steps in the Circumference turn, always keep your point free, and aim
it towards your Enemy's body. Keep your feet well together, thus the
motion in turning is smaller, andn take Measure and tempo into consideration.

\chap{Defence against Thrusts.}

% Hwar och en stöt som i rätta Mensura och tempo skeer / är intet till
% att parera eller achta sig före; Der emot när han utan dem blifwer
% giord är han på många sätt att bemöta;  Ibland alle andre är ingen
% bättre / än att du dig i ett tempo halft med kroppen och halft med
% Klingan Salverar, altså att icke allenast de blott / dijt Fiender
% stöter / utan de närmaste der hoos äro defenderade: Derföre om han
% skulle Fintera till dett ene och stöta till det andre blott / att de
% samme äro och förwarade, och wägen att råka blifwer honom förtagen /
% så lär du des hastigare stöta / och giöra din Klinga mindre
% disordinerat: Men ingen stöt skalt du slätt parera. Det wari sig en
% Slanc. stöt ;
Every thrust that happens in the right Measure and tempo, is not to be
parried or avoided; But when it is done without these it can in may
ways be countered. Among all these is none better, than that you
halfway with the body and halfway with the Blade Save, taht is not
only the opening, whereto the Enemy thursts, but also those near it
are defended. Therefore if he Feints towards one and thrusts into the
other opening, they are both defended, and the way to reach is taken
away, you can then quicker thrist, and make your Blade less
disarrayed; But no thrust should you simply parry. It should be a
counter-thrust.
%                     En sådant är tijdens tappande / whar igenom Fienden
% dig (när han sin effet muterar) stöta kan / särdeles som sagt är att
% diffesa och offesa tillijka i ett tempo på det geswindaste gå låter
% / så åft du på det sättet uthan fahra ; En hwilken är först / när
% twå tillijka röra sig den som först arriverar, den samma har wunnit: 
One such is the loss of time, wherein the Enemy you can (when he mutates
his effect) thrust, especially seen as how offence and defence both in
the same tempo is the quickest, frequently hyou can do this without
danger. Whoever is first, when two move, the first to arrive is the winner.
% Och see till at du din Klinga och Stånckel så gouvernerar, at du
% genom en juste linie, på din Wederpartz sijda / corresponderar,
% uthan serdeles movement din Arm åft defenderar, kommandes i hans
% stöt, när du tager Contratempo der emot med ditt kors / intet högre
% än hans swaga / och stöter brede wijd henne in/ så framt hans udd
% intet går utur presenza, då bör du hålla den jämbde linien i
% stöten;
And ensure that you govern your Blade and [Stånckel???], taht you with
a true line, on your Opponent's side, correspond, without large motion
can defend your Arm, coming in his thrust, when you take against it
Countertempo with your guard, no higher than his weak, and thrust
beside his blade, if his blade does not go out of present, you should
hold an even line in the thrust.
%             Stöter Adversarius diupt in / kant du wäl dett wänstra
% benet sättia tillbaka / och i samme tempo råka / såsom och med en
% hast den högre tillijka med den wänstre Schenkelen gå tillbakas /
% stöter han intet så diupt / att han kan hinna dig / så stijg med
% Contratempo och foten fram för sig.

If your Adversary thrusts deep / you can well step back with the left
leg / and in the same tempo reach / so as to with haste the right as
well as the left ankle retreat / if he does not thrust that deep /
that he can reach you / step with Counter-tempo and the foot in front
of you.

% Dee äro och somblige som giöra halfwa stöter / hwar med de söka att
% reta dig till stöta / då kan du effter hans begiäran med stöten gå
% fram för sig / dock med porterer Klinga / der med han menar hafwa
% parerat, lär du råka honom / i dett du förändrard din effect.

There are some who make half thrusts / wherein they try to tempt you
to thrust / you can in accordance with his wish / although with
[porteret???]\sidenote{Right? maybe...} Blade / as he means to have
parried, you should reach him / in that you change your effect.

% Skulle han med ett insteeg i string' Fintera eller stöta med
% styrkan innan till hos din swaga / så Cavera i dett samma / och stöt
% utan till tertiam

Should he with a step in a bind Feint or thrust with the strong inside
of your weak / Void your blade in that moment / and thrust to the
outside in Third.