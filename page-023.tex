\newpage

\scan{023}

\newpage

% Translation below

% plats / i dett Fienden will defendera den ene ypningen / geer han
% sig blott på en annan ort; Derföre är nödvändigt at wetta
% hwilkadera äro goda eller intet; En somblige giöra deras Finter med
% foter mer än med Klingan/ i det att de trampa och battera så hårdt
% på marken som de kunna / der med sökandes att skräma sin Fiende /
% hwar utaf de och få tillfälle att stöta / som ibland och låter sig
% giöra / när man hafwer att beställa med dem som intet hafwa mycket
% Courage. Men emot en som i denna konsten är öfwat och förstår
% Mensuram och tempo, låter intet giöra sig emot att taga der effter /
% såsom han wäl weet / emedan han är war så långt utur mensura, när
% han giorde Finten, som han tillförende war.

place / in that the Enemy defends one opening, he opens another
elsewhere. It is thus necessary to know which are good and not. Some
make their Feints more with their feet than with the Blade / in that
they stomp and step as hard on the ground as they can / thus trying to
frighten their enemy / which would give them an opportunity to thrust
/ as sometimes works / when you are dealing with those who have little
Courage. But against someone who is practiced in this art and
understands Measures and tempo, it is not done to act in this manner /
since he was so far out of measure, when he made the Feint, as he was.

% De äro och somblige / som giöra sine Finter i dett samma de Avancera
% med både fötterne tillika / hwilket de giöra till den ändan / att de
% desse mena skrämma sin Fiende / dett och så har en stor effect,
% särdeles emot dem som intet förstå tempo, och de som hafwa lagt sin
% defension aldeles på parader / de samma lära wäl något kunna uträtta
% / och komma så letteligen att bringa sin Fiende i confusion, emedan
% han som giorde Finten kom uti en fourie i mensuram, men der emot är
% den samma uti stor fara som så starkt Avancerar med båda fötterna i
% Finten; En om Fienden tager den tijden i acht / i dett samma den
% andre Finterar blifwer han stötter / och kan intet förswara sig; En
% då både fötterne är lösa / kan han hwarken stöta eller parera eller
% retirera sig

There are also those / who make their Feints, in the moment they
Advance with both feet / which they do to the end / that they mean to
scare their Enemy / this can also have a good effect, especially
against those who do not understand tempo, and those who have placed
their defence entirely in parries / they may be able to do something /
and may easily bring their Enemy into confusion, while he who made the
Feint came into [ fourie??? ] in measure, but against this is he in
great danger who Advances so strongly with both feet in the Feint;
Since if the Enemy takes time into account / in the moment the other
Feints he is thrust / and cannot defend himself; Since both feet are
loose / he can neither thrust not parry nor retreat.

% Utan om en Fint skall lyckas wäl / så åtte han vara giord / att
% Fienden intet annorledes kan see / än att det är en rätt stöt/ i det
% man lyfter högre foten opp fullkomligen till stöten/ och der Foten
% giör ett tempo giör handen twå rörelser Men den efftersta foten
% står fast/ så att man kan twinga sin Fiende / att han måste taga
% effter en sådan Fint.

For a Feint to succeed well / it must be made / so that the Enemy
cannot see other than / that it is a rightful thrust / in that you
lift your right foot fully / and when the Foot makes a tempo the hand
makes two movements. But the rear foot stands firm / so that one can
force one's Enemy / that he must react to such a Feint.

% Särdeles lyckas Finterne bäst när Fienden är i motü, då blifwer
% intet så wäl utaf Fienden judicerat, der de elliest äro något
% farlige/ besynnerligen när man hafwer att giöra med en/ som uti
% denna Konsten är förfaren/ hafwer en god Resolution, Förstår mens'
% och tempo wäl.

Feints succeed especially well when the Enemy is in [ motu???], then
the Enemy's judgement is compromised, when they would otherwise be
somewhat dangerous / especially when you are dealing with one / who is
in this Art well schooled / have a strong Resolve, Understands measure
and tempo well.

% De bästa och säkraste manier att Fintera, är på detta sätt:
% hwilket sker i dett du lyffter den högre foten opp/ din arm tillika
% med Klingan och kroppens öfverdehl/ såsom och hufwudet fram
% utsträcker/ hålla kroppen ett ögnebleck på den wänstra foten hwijlad
% / på dett du kan see om han parerar eller intet / hwar effter du
% rättar ditt arbete.

The best and safest manner to Feint, is this way: that which happens
when you lift your right foot up / as well as the Blade and your upper
body / and likewise stretch your head forward / hold the body resting for a
moment on the left foot / so that you can see if he parries or not /
after which you correct your work.
