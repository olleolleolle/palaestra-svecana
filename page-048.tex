\newpage

\scan{048}

\newpage

% Translation below

% Och först wara i mensura larga, och skall han wetta / att de
% öfwerste blott äro så långt och aflägse / så at han intet må
% blifwa stötter / så framt han förstår at hålla sin Klinga frij;
% Och är nu på sådant sätt denna gvardia icke utan särdeles fördel /
% och kan beqwämligen emot åtskillige angulerre som och uthsträckte
% gvardia blifwa brukat / i dett att kroppen på sådant sätt är
% Salverat och utur presenza.

And first be in wide measure, and he shall know, that the upper
openings are so far away, so that he will not be thrusted, as long as
he understands to keep his BLade free. And now in such a fashion is
this guard not without particular advantage, and can comfortably be
ised against multiple angles that extended guards can be used, in that
the body is in this fashion Saved and out of presence.

% I qvarta måtte du skränka dig något med halfwa kroppen / Klingan
% recta linea/ i den högden din Axel gvardera, så kan du lättteligen
% Cavera, och är Fienden på sådant sätt intet Affecturerat att han lär
% finna din Klinga / att du således utan till wäl förwarat /
% särdeles emot Secunda, som hon af naturen nederåht inclinerat/
I fourth you must hunch slightly with half the body / Blade in a
straight line\sidenote{recta linea''} / at the height of your Shoulder
guard, you can easily Disengage, and the Enemy is in this way not
Affected so that he easily can find your blade., that you thus on the
outside is well defended, especially against Second, which is by
nature inclined downwards,
% parera innan till / och såsom handen wänder sig dijt / må du intet
% blifwa der råkat / wharföre Fienden igenom denna gvardia blifwer wägen
% att stöta igenstängd; Wille han då taga din Klinga sinkos /
% hwliket och woro fahra hos / emädan man i denne gvard' lätteligen
% kan cavera och ricavera; Den underste blott blifwa genom Abaissering
parry on the inside, and as the hand turns towards, you will not be
reached / wherefore the Enemy through this guard the way to thrust is
closed. If he wants to take your blade bendwards, which is a danger,
since in this guard you can easily disengage and
redisengage\sidenote{Origina has 'ricavera', interpreted s a disengage
around a disengage}. The lower opening is through
Lowering\sidenote{``Abaissering'', interpreted to be analogous to 'baisse'}
% med udden eller korset (hwar efter Fienden stöter) tillijka med
% offesa defenderat; Resterar nu honom födenskull / att han dig med en
% Fint disordinerar, och passerar der på under din Klinga sinkos; En a
% pie fermo förmår han intet i denna gvard' råka; Är altså denna
% gvard' ibland alle andre uthsträckte / den bästa hwarutinnan man
% mäst sin Klinga kan hafwa liber och promper.
with the point or the cross (after the Enemy has thrusted) with
offence defended. It is now left for him, that he with a Feint
disorganises you, and pass thereafter under your blade lowered. Even
in a pie fermo he is unable to in this guard reach. Is thus this guard
among all the extended ones, the best wherein one most can have one's
Blade free and prompt.

% När du elliest i qvart något diupare gvarderar dig / öpnandes dig
% utan till / giör du bättre / om han utan till stöter / att du intet
% parerar, utan emädan din Arm på sådant sätt blifwer agnulerat, och
% hans stöt kommer innna till din Arm / lijka som man försatte med
% Armen / och qvart' med dett falska steget / utan att röra hans Klinga
% under hans arm in; Handen blifwer som hon tilförende war.

When you guard yourself somewhat deeper in fourth, you open yourself
on the outside, you would do better, if he thrusts on the outside,
taht you do not parry, but while your Arm is thus angled, and his
thrust comes inside your Arm, equally as you put your Arm, and fourth
with the false step, without touching his Blade in under his arm. The
hand remains as it was.

% Item / ligger din wederpart i secunda angulata / och du will hans
% Klinga innan till occupera / skall du giöra dett samma med qvarta
% och förskränkter kropp, din udd skall wara innan till jämpt till
% din Fiendes swaga uphögd; Leden på din högre hand måtte du wäl
% wända innan till, der med att denne angulus blifwer än mer eviterat;
% En ju mer secunda innan til är angulerat ju mera force hafwer hon
% der / och den som skulle stöta i de blott / han blefwo och sielfwer
% råkader af den angulo / så framt han intet med den vantagio rätta
% linien stötte / eller wänte på något tempo / att Fienden kommo
% närmare i Mensura / då man i det samma wänder utur qvarta i
% secunda emedan det bätre är emot secunda, när man får tillfälle at
% passera / och går man å man: Men måtte här jämpte merkia / det man
% med qvarta intet går förfort / förr än man har fått med styrkan
% Fiendens swaga / så är man för hans secunda mycket bättre Coopert.
If your opponent is in second angle\sidenote{``secunda angulata''},
and you want to occupy his Blade on the inside, you shoudl do it with
the same fourth and lowered body, your point should be in the inside,
lifted even with your Opponent's weak. The joint on your right hand
you must turn well to the inside, through this the angle will be more
elevated. The more second inwards angled, the more force the blade
will have/ and the one who would thrust in the openings, he will
himself be reached by this angle / as long as without  the right
advantageous line thrusts, or wait for the tempo, that the Enemy gets
closer in Measure, then you turn in that moment out of fourth into
second since it is better against second, when you have an opportunity
to pass, and goes mano a mano\sidenote{This... does not make sense in
the original, ``och går man å man'', literally ``and goes man and
man''}. But you should here remark, that you do not with fourth go too
fast, before one has caught with the string the Enemy's weak, you are
for his Second much better enveloped[???]\sidenote{``Coopert'', could
be a corrupted ``copert'', which is an old spelling of ``kuvert'',
which means envelope, as in letter}